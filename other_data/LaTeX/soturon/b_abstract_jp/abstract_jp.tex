\documentclass[report, a4paper,11pt]{../thesis}
\usepackage[dvips]{graphicx}
\usepackage{ascmac} %枠で囲むやつら

%\renewcommand{\thesection}{\arabic{chapter}.\arabic{section}}
%\renewcommand{\thesubsection}{\arabic{chapter}.\arabic{section}.\arabic{subsection}}

\makeatletter       %参考文献を[1]から(1)へさらにciteを上付きへ
  \renewcommand{\@biblabel}[1]{(#1)}
  \DeclareRobustCommand\cite{\unskip
  \@ifnextchar[{\@tempswatrue\@citex}{\@tempswafalse\@citex[]}}
  \def\@cite#1#2{$\!^{\hbox{\scriptsize{(#1\if@tempswa , #2\fi})}}$}
  \def\@biblabel#1{#1)}
  \@addtoreset{figure}{chapter}
\makeatother

\addtolength{\oddsidemargin}{31truemm}
\addtolength{\textwidth}{-55truemm}
\setlength{\fullwidth}{\textwidth} 
%\addtolength{\topmargin}{-1truein}
\setlength{\headheight}{0pt}
\setlength{\headsep}{0pt}
\addtolength{\textheight}{20truemm}
\renewcommand{\baselinestretch}{1.2}
\newcommand{\figref}[1]{Fig:\ref{#1}}
\newcommand{\tableref}[1]{Table \ref{#1}}

\begin{document}

\pagestyle{empty}
% -------------------------------------
%  和文要旨
% -------------------------------------
\newpage
\begin{center}
{\Large ティルトロータ型UAVの低速域での飛行モデル構築\\ \vspace{5mm} 弓場 洋輝} \vspace{10mm}
\end{center}

{\Large {\bf 要旨}\vspace*{10mm}}

本研究では,大規模災害時における情報収集を目的としたティルトロータ型UAV(Unmanned Aerial Vehicle)の低速域における飛行モデルの構築を行なう.対象とするUAVは,同軸二重反転構造のメインロータを有しており,このメインロータをティルトさせることにより,ヘリコプタモード,遷移モード,飛行機モードの3つの飛行モードの切り替えを行なう.ヘリコプタモードでは,機体中央部のメインロータだけでなく,機体の左右翼と機首部にあるサブロータを用いて,浮上揚力と姿勢制御トルクを発生させ,ホバリング飛行を行なう.本研究では,このヘリコプタモードを対象としている.まず,ホバリング時における空気力を考慮した飛行モデルを導出する.機体の運動を縦方向と横方向とに分け,低次元化したモデルを構築する.次に,開発した機体を用いて,マニュアル操縦での飛行実験を行ない,得られた飛行データから飛行モデルにおける未知パラメータの推定を行なう.主に,このパラメータ推定の精度を上げるための入出力データの処理方法について検証する.

\end{document}
