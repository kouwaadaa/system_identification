% 第3章
\ref{problem}章で述べた水系運用計画最適化問題の記述に基づき,数理計画モデルを構築する.
まず,問題の基本要素と属性を表す記号を導入した上で,目的関数および制約条件を%
形式的に記述する.
\section{基本要素}
水系運用計画最適化問題の基本要素とその属性を示す.
なお,以下で各要素の先頭に付した記号は
$\star$\, : 決定変数,
$\circ$\, : 従属変数,
$\bullet$\, : 定数,
をそれぞれ意味する.

\begin{itemize}
\item[(1)]
	ダム $ \mathrm{D}_i $ $(i \in \mathcal{D},ただし\mathcal{D}はダムの集合) $
	%ダム $ \mathrm{D}_i $ $(i \in \mathcal{D} = \{1,\hdots,n^{\rm D}\}) $
	\begin{itemize}
		\item[$\circ$] 
			貯水量 : $s_i(k)$
		\item[$\bullet$] 
			許容貯水量の上限値 : $s_{i}^{\mathrm{Max}}$
		\item[$\bullet$] 
			許容貯水量の下限値 : $s_{i}^{\mathrm{Min}}$
		\item[$\bullet$] 
			最終貯水量の上限値 : $s_{i}^{\mathrm{Last,Max}}$
		\item[$\bullet$] 
			最終貯水量の下限値 : $s_{i}^{\mathrm{Last,Min}}$
		\item[$\bullet$]  
			渓流量 : $a_i(k)$
%	\item[$\bullet$] 
%		$\tilde{z}_{i}^{\rm D}$ : 均等放流フラグ(均等放流が必要ならば1,そうでなければ0)
%	\item[$\bullet$] 
%		放流定常量 : $\tilde{q}_{i}^{\rm D}$ : 放流定常量[$\mathrm m^3$]%
%	\item[$\bullet$] 
%		$\tilde{ \mathcal C}_{i}^{\rm G}$ : 均等放流を行なう発電放流路の集合
%	\item[$\bullet$] 
%		$\tilde{ \mathcal C}_{i}^{\rm F}$ : 均等放流を行なうゲート放流路の集合
	\end{itemize}
	ただし,
	\begin{itemize}
		\item 貯水量 : ダム$i$の第$k$時間帯初頭における貯水量[$\mathrm m^3$].
		\item 貯水量上限値 : ダム$i$の許容貯水量の上限値[$\mathrm m^3$].
		\item 貯水量下限値 : ダム$i$の許容貯水量の下限値[$\mathrm m^3$].
		\item 最終貯水量上限値 : ダム$i$の貯水量の最終値の上限値[$\mathrm m^3$].
		\item 最終貯水量下限値 : ダム$i$の貯水量の最終値の下限値[$\mathrm m^3$].
		\item 渓流量 : 第$k$時間帯にダム$i$に流入する雨水などの水量[$\mathrm m^3$].
	\end{itemize}
%
\item[(2)]
	発電放流路 $\mathrm{G}_{ii'j}$ $( (i,i') \in \mathcal {P}^{\rm G} ,j \in \mathcal {J}_{ii'} , %
	ただし \mathcal {P}^{\rm G}は発電機が存在する放流路(ダム対)の集合, %
	また\mathcal{J}_{ii'}は放流路ii'に存在する発電機の集合) $
%
	\begin{itemize}
		\item[$\star$] 
			発電放流量 : $q^{\rm G}_{ii'j} (k)$
		\item[$\circ$] 
			発電量 : $y^{\rm G}_{ii'j}(k)$
		\item[$\bullet$]
			流下時間帯数(上部): $\tau_{ii'j}^{ \rm {G,Up} }$
		\item[$\bullet$]
			流下時間帯数(下部): $\tau_{ii'j}^{ \rm {G,Down} }$
		\item[$\bullet$] 
			最大水量 : $q_{ii'j}^{\mathrm{G,Max}}$ 
		\item[$\bullet$] 
			最小水量 : $q_{ii'j}^{\mathrm{G,Min}}$
		\item[$\bullet$] 
			電水比 : $\alpha_{ii'j}$
		\item[$\circ$] 
			運転フラグ : $z^{\rm G}_{ii'j}(k) \in \{0,1\}$
		\item[$\star$] 
			停止フラグ : $z_{ii'j}^{\mathrm {G,OFF}}(k) \in \{0,1\} $
		\item[$\star$] 
			起動フラグ : $z_{ii'j}^{\mathrm{G,ON}}(k) \in \{0,1\} $
%	\item[$\bullet$] 
%		$\underline{z}^{\rm G}_{ii'j}(k)$ : 第$k$時間帯の運転状況($ k \leq 0 $)
%	\item[$\bullet$] 
%		$\underline{z}_{ii'j}^{\mathrm {G,OFF}}(k)$ : 停止フラグ($ k \leq 0 $)
%	\item[$\bullet$] 
%		$\underline{z}_{ii'j}^{\mathrm{G,ON}}(k)$ : 起動フラグ($ k \leq 0 $)	
		\item[$\bullet$] 
			停止期間 : $\tau^{\mathrm{G,OFF}}_{ii'j}$
		\item[$\bullet$] 
			運転期間 : $\tau^{\mathrm{G,ON}}_{ii'j}$
		\item[$\bullet$]
			計画停止期間 : $\mathcal{K}^{\rm S}_{ii'j}$
		\item[$\bullet$]
			計画運転期間 : $\mathcal{K}^{\rm R}_{ii'j}$
	\end{itemize}
	ただし,
	\begin{itemize}
		\item 発電放流量 : 第$k$時間帯に放流される水量[$\mathrm m^3$]%
		($ k \leq 0 については定数として与えられているものとする$).
		\item 発電量 : 第$k$時間帯に発電される発電電力量[kWh].
		\item 流下時間帯数(上部): 上流ダムから放流した水が%
			発電機に到達するのにかかる時間帯数.
		\item 流下時間帯数(下部): 発電機から流下した水が%
			下流ダムに到達するのにかかる時間帯数.
		\item 最大水量 : 発電に使用可能な最大水量[$\mathrm m^3$].
		\item 最小水量 : 発電に使用可能な最小水量[$\mathrm m^3$].
		\item 電水比 : 発電電力量と使用水量の比率[$\mathrm {kWh}/ \mathrm m^3$].
		\item 運転フラグ : 発電機が第$k$時間帯に運転していれば1,停止していれば0となる%
		($ k \leq 0 については定数として与えられているものとする$).
		\item 停止フラグ  : 第$k$時間帯初頭(第$(k-1)$時間帯末尾)に発電機を停止するとき1,%
			そうでないとき0となる%
			($ k \leq 0 については定数として与えられているものとする$).
		\item 起動フラグ  : 第$k$時間帯初頭(第$(k-1)$時間帯末尾)に発電機を起動するとき1,%
			そうでないとき0となる%
			($ k \leq 0 については定数として与えられているものとする$).
		\item 停止期間 : 発電機を停止させた際に,最低限停止させておくべき時間帯数.
		\item 運転期間 : 発電機を起動させた際に,最低限運転を継続させるべき時間帯数.
		\item 計画停止期間 : 発電機を停止させる時間帯.
		\item 計画運転期間 : 発電機を起動させる時間帯.
	\end{itemize}
%
\item[(3)]
	ゲート放流路 $\mathrm{F}_{ii'}$ $ ((i,i') \in \mathcal{P}^{\rm F} , %
	ただし \mathcal{P}^{\rm F} はゲート放流路(ダム対)の集合) $
%
	\begin{itemize}
	\item[$\star$] 
		ゲート放流量 : $q^{\rm F}_{ii'}(k)$
	\item[$\bullet$]
		放流量上限値 : $q^{\mathrm {F,Max}}_{ii'}$
	\item[$\bullet$]
		流下時間帯数 : $\tau_{ii'}^{ \rm F }$
	\end{itemize}
	ただし,
	\begin{itemize}
		\item ゲート放流量 : 第$k$時間帯に放流される水量$[\mathrm m^{3}]$.
		\item 放流量上限値 : ゲート放流量の上限値$[\mathrm m^{3}]$.
		\item 流下時間帯数 : 上流ダムから放流した水が%
			下流ダムに到達するのにかかる時間帯数.
	\end{itemize}
%
\item[(4)]
	バイパス放流路 $\mathrm{B}_{ii'}$ $ ( (i,i') \in \mathcal{P}^{\mathrm B} , %
	ただし \mathcal {P}^{\mathrm B} はバイパス放流路(ダム対)の集合) $
%
	\begin{itemize}
		\item[$\bullet$]
			放流量 : $q^{\mathrm {B}}_{ii'}(k)$
		\item[$\bullet$]
			流下時間 : $\tau_{ii'}^{ \mathrm{B} }$
	\end{itemize}
	ここで,
	\begin{itemize}
		\item 放流量 : 第$k$時間帯に放流される水量$[\mathrm m^3]$.
		\item 流下時間帯数 : 上流ダムから放流した水が%
			下流ダムに到達するのにかかる時間帯数.
	\end{itemize}
%
\item[(5)]
	スイッチ放流路 $ \mathrm{Y}_{ii'}$ $ ( (i,i') \in \mathcal{P}^{\mathrm Y} , %
	ただし \mathcal {P}^{\mathrm Y} はスイッチ放流路(ダム対)の集合) $
%
	\begin{itemize}
		\item[$ \circ$]
			スイッチ放流量 : $q^{\mathrm {Y}}_{ii'}(k)$
		\item[$ \bullet$]
			放流量上限値 : $q^{\mathrm {Y,Max}}_{ii'}$
		\item[$ \bullet$]
			流下時間帯数 : $ \tau_{ii'}^{\mathrm Y}$
		\item[$ \bullet$]	
			対応発電放流路群 : $ \mathcal C_{ii'}^{\rm D}$
		\item[$ \bullet$]	
			分流条件 : $ c_{ii'}^{\rm S}$
		\item[$ \bullet$]
			流下時間帯数(発電機): $\tau_{ii'}$
	\end{itemize}
	ただし,
	\begin{itemize}
		\item 放流量上限値 : スイッチ放流量の上限値$[\mathrm m^3]$.
		\item 流下時間帯数 : 上流ダムから放流した水が%
			下流ダムに到達するのにかかる時間帯数.
		\item 対応発電放流路群 : スイッチ放流量に関する条件が規定される発電放流路群
		\item 分流条件 : 対応発電放流路群中の発電放流路が %
			1つでも運転しているときにスイッチ放流量が0となるとき0となり, %
			すべて停止しているときに0となるとき1となる.
		\item 流下時間帯数(発電機): 上流ダムから放流した水が%
			対応する発電放流路に到達するのにかかる時間帯数.
	\end{itemize}
%
\item[(6)]
	特別放流路 $ \mathrm{X}_{ii'}$ $ ( (i,i') \in \mathcal{P}^{\mathrm X} , %
	ただし \mathcal {P}^{\mathrm X} は特別放流路(ダム対)の集合) $
%
	\begin{itemize}
		\item[$ \circ$]
			放流量 : $q^{\mathrm {X}}_{ii'}(k)$
		\item[$ \bullet$]
			放流量上限値 : $q^{\mathrm {X,Max}}_{ii'}$
		\item[$ \bullet$]
			流下時間帯数 : $ \tau_{ii'}^{\mathrm X}$
	\end{itemize}
	ただし,
	\begin{itemize}
		\item 特別放流路 : 第$k$時間帯に放流される水量$[\mathrm m^3]$.
		\item 放流量上限値 : 特別放流量の上限値$[\mathrm m^3]$.
		\item 流下時間帯数 : 上流ダムから放流した水が%
			下流ダムに到達するのにかかる時間帯数.
	\end{itemize}
%
\item[(7)]
	その他 
%
	\begin{itemize}
		\item[$\bullet$] 
			時間帯 : $ \mathcal{K} = \{ 1 , \hdots , n^{\rm K} \} $
		\item[$\bullet$]
			時間帯(過去) : $\mathcal{K}^{\rm P} = \{ 0 , -1 , \hdots , n^{\rm P} \} $
		\item[$\bullet$] 
			時間帯(夜間): $\mathcal {K}^{\rm N}$
		\item[$\bullet$] 
			発電価値 : $\varphi (k)$
%	\item[$\bullet$]
%		$\mathcal{D}^{\rm E}$ : 均等放流が必要なダムの集合
		\item[$\bullet$]
			発電所 : $\mathcal{G}_{n}$($n \in \cal{N}$ , ただし$\cal N$は%
			後述するALL制約の対象となる発電放流路群の集合)
%	\item[$\bullet$] 
%		稼働段階 : $\mathcal{M}_{n}$ : 発電放流路群がもつ稼働段階の集合 %
%		($\mathcal{M}_{n}$ = \{ 1 , 2 , $\hdots$ , $|\mathcal{M}_{n}|$ \} )
		\item[$\bullet$]
			段階的最大水量 : $\hat{q}^{m}_{n} $
		\item[$\circ$]
			稼働段階 : $\hat{z}^{m}_{n} (k) \in \{ 0 , 1 \}$%
			($m \in \mathcal M_{n} = \{1,\hdots,|\mathcal M_{n}|$)
		\item[$\bullet$]
			待機時間 : $\hat{\tau}^{m}_{n}$
	\end{itemize}
	ここで,
	\begin{itemize}
		\item 時間帯 : 水系運用の対象となる期間.
		\item 時間帯(過去): 過去の放流実績などが含まれる時間帯.
		\item 時間帯(夜間): 夜間に含まれる時間帯.
		\item 発電価値 : 第$k$時間帯における単位量当たりの発電価値[1/kWh].
		\item 発電所 : 後述するALL制約の対象となる発電放流路の組.
		\item 段階的最大水量 : 発電所$n$において第$m$段階で発電に使用可能な%
		最大水量$[\mathrm m^3]$.
		\item 稼働段階 : 第$k$時間帯の発電機の稼働段階.第$m$段階のとき1となり,%
		そうでないとき0となる.
		\item 待機時間 : 発電機の稼働段階が第m段階から次の段階に移るまでの時間帯数.
	\end{itemize}	
\end{itemize}	
	
%%%%%%%%%%%	
\section{制約条件}
%%%%%%%%%%%

制約条件を以下に示す.
%
	\begin{itemize}
%
	\item 貯水量制約 \\
	 ダム貯水量が許容範囲内になければならない.	
%
		\begin{equation}
			s_{i}^{\mathrm{Min}} \leq s_i(k) \leq s_{i}^{\mathrm{Max}} %
			\hfill ({}^{\forall} i \in \mathcal {D} , {}^{\forall} k \in \mathcal{K})
		\end{equation}
%
	\item 最終貯水量制約 \\
	 ダム貯水量の最終値が許容範囲内になければならない.
%
		\begin{equation}
			s_{i}^{\mathrm{Last,Min}} \leq s_{i}( |\mathcal{K}| + 1 ) %
			\leq s_{i}^{\mathrm{Last,Max}} %
			\hfill ({}^{\forall} i \in \mathcal {D} )
		\end{equation}
%
	\item 
		貯水量の推移式 \\
		 第$k$時間帯に$\mathrm D_i$から流出する水量の総和を$O_{i}(k)$,%
		流入する水量の総和を$I_{i}(k)$とすると,%
		$\mathrm D_i$の貯水量は次式に従って推移することになる.
%
		\begin{equation}
			s_i(k+1) = s_i(k) + a_i(k) - O_{i} (k) + I_{i} (k)
			\hfill ({}^{\forall} i \in \mathcal {D} , {}^{\forall} k \in \mathcal{K})
		\end{equation}		
%
		ここで,$\mathrm D_i$の直前にあるダム群および直後にあるダム群%
		を,それぞれ $\mathcal{P}_i$ および $\mathcal{S}_i$とすると,%
		$O_{i}(k)$と$I_{i}(k)$は以下のようになる.
%
		\begin{equation}
		\begin{split}
			O_{i}(k) = \sum_{ j \in \mathcal J_{ii'} } \sum_{ i' \in \mathcal S_i} 
			\{ q^{\rm G}_{ii'j} (k)  +q^{\rm F}_{ii'}(k) + q^{\rm B}_{ii'} (k) + q^{\rm X}_{ii'} (k)
			+ q^{\rm Y}_{ii'} (k) \} 
		\end{split}
		\end{equation}
%
		\begin{equation}
		\begin{split}
			I_{i}(k) = & \sum _{ j \in \mathcal{J}_{i'i} } \sum_{i' \in \mathcal P_i} 
		 	\{ q^{\rm G}_{i'ij} ( k- \tau_{i'ij}^{ \rm{G,Up} } - \tau_{i'ij}^{ \rm{G,Down} } )
		 	+ q^{\rm F}_{i'i} ( k- \tau_{ii'}^{ \rm F } ) \\
			&+ q^{\rm B}_{i'i} ( k - \tau_{i'i}^{ \rm B })
			+ q^{\rm X}_{i'i} ( k -\tau_{i'i}^{\rm X})
			+ q^{\rm Y}_{i'i} ( k -\tau_{i'i}^{\rm Y}) \} 
		\end{split}
		\end{equation}			
%	
	\item 水量制約 \\
	 発電機の運転時には使用水量が許容範囲内になければならない.
%
		\begin{equation}
			z^{\rm G}_{ii'j} (k) q_{ii'j}^{\mathrm{G,Min}} \leq q^{\rm G}_{ii'j} (k) \leq %
			z^{\rm G}_{ii'j} (k) q_{ii'j}^{\mathrm{G,Max}} 
		\end{equation}
%	
		\hspace{7.5cm}
		$ ( {}^{\forall} j \in \mathcal{J}_{ii'} , {}^\forall(i,i') \in \mathcal{P}^{\rm G} , %
		{}^{\forall} k \in \mathcal{K}) $
%
	\item 電水比制約 \\
	 電水比により,発電機の使用水量と発電電力量が規定される.
%
		\begin{equation}
			y^{\rm G}_{ii'j}(k) = \alpha_{ii'j} q^{\rm G}_{ii'j}(k - \tau^{\rm G,Up}_{ii'j})%
			\hfill ( {}^{\forall} j \in \mathcal{J}_{ii'} , {}^\forall(i,i') \in \mathcal{P}^{\rm G} , %
			{}^{\forall} k \in \mathcal{K}) 
		\end{equation}	
%		
	\item 計画運転・停止制約 \\
	 計画運転・停止期間において発電機は運転,または停止状態になければならない.
%
		\begin{equation}
			z^{\rm G}_{ii'j} (k) = 1
			\hfill ( {}^{\forall} j \in \mathcal{J}_{ii'} , {}^\forall(i,i') \in \mathcal{P}^{\rm G} , %
			{}^{\forall} k \in \mathcal{K}^{\rm R}_{ii'j})
		\end{equation}	
%		
		\begin{equation}
			z^{\rm G}_{ii'j} (k) = 0
			\hfill ( {}^{\forall} j \in \mathcal{J}_{ii'} , {}^\forall(i,i') \in \mathcal{P}^{\rm G} , %
			{}^{\forall} k \in \mathcal{K}^{\rm S}_{ii'j})
		\end{equation}	
%		
	\item
		短時間運転・停止制約 \\
		 発電機を運転させると,期間$\tau^{\rm G,ON}_{ii'j}$は運転を継続する必要が%
		ある.また,停止させると期間$\tau^{\rm G,OFF}_{ii'j}$の間を空ける必要がある.	
%
		\begin{equation}
			z^{\rm G}_{ii'j} ( k' ) \geq z_{ii'j}^{\mathrm {G,ON}} (k)
		\end{equation}	
%
		\begin{equation}
		\begin{split}
			( {}^{\forall} j \in \mathcal{J}_{ii'} , {}^\forall(i,i') \in \mathcal{P}^{\rm G} , %
			{}^{\forall} k \in \mathcal{K} \cup \mathcal{K}^{\rm P} ,%
			k' \in \{ k , \hdots , k + \tau^{\rm G,ON}_{ii'j} - 1 \} ) \nonumber
		\end{split}	
		\end{equation}
%		
		\begin{equation}
			z^{\rm G}_{ii'j}(k') \leq 1 - z_{ii'j}^{\mathrm {G,ON}} (k) 
		\end{equation}	
%		
		\begin{equation}
		\begin{split}
			( {}^{\forall} j \in \mathcal{J}_{ii'} , {}^\forall(i,i') \in \mathcal{P}^{\rm G} , %
			{}^{\forall} k \in \mathcal{K} \cup \mathcal{K}^{\rm P} ,
			k' \in \{ k , \hdots , k + \tau^{\rm G,OFF}_{ii'j} - 1 \} ) \nonumber
		\end{split}	
		\end{equation}	
%		
		ここで,各フラグにはTable. \ref{rof}に示す関係が成り立つ.
		\begin{table}[tbp]
		\begin{center}
		\caption{Relation of flags}
		\label{rof}
		\begin{tabular}{lllllll}
		\hline
 			& $z^{\rm G}_{ii'j}(k-1)$ & $z^{\rm G}_{ii'j}(k)$ & $z_{ii'j}^{\mathrm {G,ON}} (k)$ & $z_{ii'j}^{\mathrm {G,OFF}} (k)$\\ 
		\hline	
			Start up & 0 & 1 & 1 & 0\\
			Stop & 1 & 0 & 0 & 1\\
			Continuation of stop & 0 & 0 & 0 & 0\\
			Continuation of run & 1 & 1 & 0 & 0\\
		\hline	
		\end{tabular}
		\label{relation}
		\end{center}
		\end{table}				
	\\
%
%		 ここで,$z^{\rm G}_{ii'j}(k),z^{\rm G}_{ii'j}(k-1),z_{ii'j}^{\mathrm {G,ON}} (k)$の間に%
%		次の関係が成り立つ.
%
		$z^{\rm G}_{ii'j}(k),z^{\rm G}_{ii'j}(k-1),z_{ii'j}^{\mathrm {G,ON}} (k)$について,%
		この関係を線形の式で表す.
		\begin{equation}
			 z_{ii'j}^{\mathrm {G,ON}} (k) \geq z^{\rm G}_{ii'j}(k) - z^{\rm G}_{ii'j}(k-1)
		\end{equation}
%		
		\begin{equation}	 
			z_{ii'j}^{\mathrm {G,ON}} (k) \leq \frac{z^{\rm G}_{ii'j}(k) - z^{\rm G}_{ii'j}(k-1)+1}{2}
		\end{equation}
%
%		 同様に,$z^{\rm G}_{ii'j}(k),z^{\rm G}_{ii'j}(k-1),z_{ii'j}^{\mathrm {G,OFF}} (k)$%
%		の間に次の関係が成り立つ.
%
		同様に,$z^{\rm G}_{ii'j}(k),z^{\rm G}_{ii'j}(k-1),z_{ii'j}^{\mathrm {G,OFF}} (k)$に%
		ついても,線形の関係式で表す.
		\begin{equation}
			z_{ii'j}^{\mathrm {G,OFF}} (k) \geq z^{\rm G}_{ii'j}(k-1) - z^{\rm G}_{ii'j}(k)
		\end{equation}	
%		
		\begin{equation}
			z_{ii'j}^{\mathrm {G,OFF}} (k) \leq %
			\frac{z^{\rm G}_{ii'j}(k-1) - z^{\rm G}_{ii'j}(k)+1}{2}
		\end{equation}
%
		%$z_{ii'j}^{\mathrm {G,ON}} (k)$と$z_{ii'j}^{\mathrm {G,OFF}} (k)$は
		発電機の起動と停止が同時に行なわれることはないので,	
%
		\begin{equation}
			z_{ii'j}^{\mathrm {G,ON}} (k) + z_{ii'j}^{\mathrm {G,OFF}} (k) \leq 1%
		\end{equation}
%		
		\hspace{6.3cm}
		$ ({}^{\forall} j \in \mathcal{J}_{ii'} , %
		 {}^\forall(i,i') \in \mathcal{P}^{\rm G} , {}^{\forall} k \in \mathcal{K} %
		 \cup \mathcal{K}^{\rm P}) $
%
	\item
		ALL制約 \\
		 発電所(特定の発電機の組)に使用可能な最大水量について,%
		段階的に最大水量が上昇する.\\%
%
		 発電所へ使用する水量は特定の発電放流路への放流量の和として表される.
%
		\begin{equation}
			\label{all1}
			\hat{q}^{\mathrm G}_{n} (k) = \sum_{( (i,i'),j) \in \mathrm {G}_{n}} %
			q^{\mathrm G}_{ii'j} (k)
			\hfill ({}^{\forall} n \in \mathcal{N} , {}^{\forall} k \in \mathcal{K}%
			\cup \mathcal{K}^{\rm P}) 			
		\end{equation}
%
		 発電所へ使用可能な最大水量は稼働段階に応じた値になる.
%			
		\begin{equation}
			\label{all2}
			\hat{q}^{\mathrm G}_{n} (k) \leq 
			\sum_{m \in \mathcal{M}_{n}} \hat{q}^{m}_{n}% 
			\hat{z}^{m}_{n} (k)
			\hfill ({}^{\forall} n \in \mathcal{N} , %
			{}^{\forall} k \in \mathcal{K} \cup \mathcal{K}^{\rm P}) 			
		\end{equation}
%
		 稼働段階はいずれか1つのみが1になる.		
%		
		\begin{equation}
			\label{all3}
			\sum_{m \in \mathcal{M}_{n}} \hat{z}^{m}_{n} (k) = 1 
			\hfill ({}^{\forall} n \in \mathcal{N} , m \in \mathcal{M}_{n} , 
			{}^{\forall} k \in \mathcal{K} \cup \mathcal{K}^{\rm P}) 
		\end{equation}
%
		 段階$m$にある発電所への放流量が最大水量$\hat{q}^{m}_{n}$に達したとき,%
		待機時間$\hat{\tau}^{m}_{n}$をおいて段階$m+1$に移ることができる.
%
		\begin{equation}
			\label{all4}
			\hat q^{\rm G}_{n} ( k - \hat{\tau}^{m}_{n} )% 
			\geq % 
			\hat{q}^{m}_{n} \hat{z}^{m+1}_{n} ( k )
			\hfill ({}^{\forall} n \in \mathcal{N} , m \in \mathcal{M}_{n} , 
			{}^{\forall} k \in \mathcal{K} \cup \mathcal{K}^{\rm P}) 
		\end{equation}	
%
	\item 
		夜間出力増制約 \\
		 夜間の時間帯では発電機の出力を増加させてはならない.
%
		\begin{equation}
			y_{ii'j}(k-1) \geq y^{\rm G}_{ii'j}(k) %
			\hfill ({}^\forall(i,i') \in \mathcal{P}^{\rm G} , {}^{\forall} j \in \mathcal{J}_{ii'} , %
			{}^{\forall}k \in \mathcal{K}^{\rm N})
		\end{equation}	 
%
	\item 放流量制約 \\
	 それぞれの放流量は許容範囲内になければならない.			
		\begin{equation}
			0 \leq q^{\rm F}_{ii'}(k) \leq q_{ii'}^{\mathrm{F,Max}} 
			\hfill ({}^\forall(i,i') \in \mathcal{P}^{\rm F} , {}^{\forall} k \in \mathcal{K})
		\end{equation}	
		\begin{equation}
			0 \leq q^{\rm Y}_{ii'} (k) \leq q^{\rm {Y,Max}}_{ii'}
			\hfill ({}^\forall(i,i') \in \mathcal{P}^{\rm Y} , {}^{\forall} k \in \mathcal{K})
		\end{equation}
		\begin{equation}
			0 \leq q^{\rm X}_{ii'} (k) \leq q^{\rm {X,Max}}_{ii'} \\%
			\hfill ({}^\forall(i,i') \in , \mathcal{P}^{\rm X}  , {}^{\forall} k \in \mathcal{K}) 
		\end{equation}
%
	\item
		分流制約 \\
		 放流路$ll'$に存在する発電機の運転状況に応じて,$\mathrm D_{i}$で%
		水の流れを切り替える.対応する発電機群が1つでも運転していれば, %
		分流条件が1のスイッチ放流路に放流を行ない,%
		運転しているものがなければ分流条件が0のスイッチ放流路に放流を行なう.\\
		 分流条件が1のスイッチ放流路の放流量への制約は次式のようになる.
%
		\begin{equation}
			q^{\rm X}_{ii'} (k) \leq %
			q_{ii'}^{\rm X,Max} \sum_{ (( l , l' ) , j ) \in \mathcal C_{ii'}^{\rm D}} %
			{z}^{\rm G}_{ll'j} (k+ \tau_{ii'})
		\end{equation}
		\hspace{6.3cm}
			$ ( \{ ( i , i' ) | c_{ii'}^{\rm S} =1 , ( i , i' ) \in \mathcal {P}^{\rm Y}\} %
			, {}^{\forall} k \in \mathcal {K}) $ \\
%		
		 分流条件が0のスイッチ放流路の放流量への制約は次式のようになる.
%		
		\begin{equation}
			q^{\rm X}_{ii'} (k) \leq ( 1 - z^{\rm G}_{ll'j} (k + \tau_{ii'}) ) q_{ii'}^{\rm X,Max}
		\end{equation}
%
		\hspace{3.6cm}
		$ ( \{ ( i , i' ) | c_{ii'}^{\rm S} = 0 , ( i , i' ) \in \mathcal {P}^{\rm Y} \} %
		 , (( l , l' ) , j ) \in \mathcal C_{ii'}^{\rm D} , {}^{\forall} k \in \mathcal {K}) $		

%
%	\item 
%		均等放流制約 : 特定のダムにおいては,24時間定常的に放流しなければならない %
		%($ i \in \mathcal{D}^{\rm E}_{i} , i' \in \mathcal {S}_{i} , ただし,\mathcal{D}^{\rm E}は
		%均等放流が必要なダムの集合,\mathcal {S}_{i}は\mathcal {D}^{ \rm E}_{i}の %
		%直後のダムの集合とする$).
%
%		\begin{equation}
%			\sum_{ ( ( i , i' ) , j ) \in \tilde { \mathcal {C}}_{i}^{\rm G} } q^{\rm G}_{ii'j} (k) %
%			+ \sum_{ ( ( i , i' ) , j ) \in \tilde { \mathcal {C}}_{i}^{\rm F} } q^{\rm F}_{ii'}(k) = %
%			\tilde{q}_{i}^{D} %
%			\hfill ( i \in \mathcal{D}^{\rm E} , {}^{\forall} k \in \mathcal {K} )
%		\end{equation}
%
%	\item
%		【補足】
%
%		\begin{itemize}
%			\item  式(\ref{all2})はある時間帯$k$における発電放流量$q^{\rm G}_{n} (k)$の%
%			上限値が稼働状態に応じた値になることを表す.
%			\item 式(\ref{all3})は発電機の稼働状態$\hat{z}^{m}_{ii'j} (k)$が%
%			いずれか1つ1になることを表している.
%			\item 式(\ref{all4})は第$k$時間帯の発電放流量$q^{\rm G}_{n} (k)$が%
%			第$m$段階の上限値$\hat{q}^{m}_{n}$であったときに,%
%			待機時間$\hat{\tau}^{m}_{n}$をおいて,第$(m+1)$段階へ移ることが%
%			できることを表す.
%		\end{itemize}
%		
	\end{itemize}
	
%%%%%%%%%%
\section{目的関数}
%%%%%%%%%%

	発電価値の総和最大化とゲート放流量の総和最小化を統合したものを用いる.
		\begin{equation}
		\begin{split}
			\mathrm {max} \ \ \ \ 
			z = \sum_{ j \in \mathcal {J}_{ii'}}  \sum_{ ( i , i' ) \in \mathcal {P}^{\mathrm G}} %
			\sum_{k \in \mathcal{K}} \varphi (k) y^{\mathrm G}_{ii'j}(k)%
			- w \sum_{ ( i ,  i' ) \in \mathcal{P}^{\mathrm F}}%
			\sum_{k \in \mathcal{K}} q^{\mathrm F}_{ii'} (k) 
		\end{split}	
		\end{equation}
	ここで,$w$はゲート放流量の総和に応じたペナルティの重み係数である.
	
%%%%%%%%%	
\section{まとめ}
	本章では,\ref{problem}章の水系運用計画最適化問題の定義に基づき,
	数理計画モデルの定式化を行なった.
	次章では,計算例を用いて,このモデルの妥当性について検討する.
%%%%%%%%%