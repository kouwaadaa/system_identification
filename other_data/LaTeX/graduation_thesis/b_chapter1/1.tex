% 第1章 序論

地震や津波などの自然現象による災害発生時,大規模で広範囲に及ぶ被災地での救助活動において,正確な情報収集を迅速かつ安全に行なうことが必要とされる.被災地では,建物の崩壊や地盤沈下などの理由による交通網の混乱で,地上における情報収集活動は困難となる.同時に,緊急を要する救援物資の運搬も難しくなる.

これらの課題に対し,地上状態の影響を受けない空の利用が有効であり,特に,無人航空機(Unmanned Aerial Veheicle, UAV)は有用である.UAVは,その名前のとおり操縦者を機体に搭乗しない航空機であり,大規模災害時に有人操縦者が行なうには危険な任務を遂行することが可能である.災害調査や空撮を行ない,荷物の積載能力や離着陸能力によっては,運搬利用も可能である.ただし物資運搬で利用する場合は,救助者の近くを飛行するため対人安全性も考慮しなければならない.

UAVは,飛行機のような形状の固定翼機と,ヘリコプタのようなロータ(回転翼)を持つ回転翼機に大きく分けられ,それぞれ異なる特徴を持つ.固定翼機は,速い巡航速度で飛行でき,推進効率も良いため長距離の飛行が可能であるが,一方で離着陸に滑走路が必要である.回転翼機は,垂直離着陸で,ホバリングも可能であるが,一方で固定翼機に比べて巡航速度が遅く,推進効率が悪い.

現在,災害発生時にUAVが利用される場合,それぞれの長所と短所を考慮し,災害状況や用途によって使い分けられている.しかし,大規模災害発生時においては,従来のUAVでは任務を効率的に遂行することができないため,より高い性能を持つUAVの開発が望まれる.

そこで我々は,固定翼機と回転翼機のそれぞれの長所をあわせ持つ,ティルトロータ機に着目した.ティルトロータ機とは,推力を発生させるメインロータを機体に対して鉛直方向から水平方向まで可動させることで,固定翼機モードと回転翼機モードを切り替えることができる航空機である.このようなティルトロータを有したUAVは,大規模災害発生時の情報収集に適した機体であると考えられる.本研究で使用するUAV(以下開発機体と表記)は,エアロセンス株式会社と共同で開発を行なったものである.軽量かつ剛性の高い,対人安全性を考慮した独自形状機構を持つ,ティルトロータを有した小型UAVである.機体の詳細は第2章で述べる.

本研究では,開発機体を対象とし,特に回転翼機モード時の飛行特性解析を行なう.
