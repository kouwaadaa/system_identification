% 第5章 低速飛行特性
%!TEX root = main.tex

%%%%%%%%%%%%%%%%%%%%%%
\chapter{低速飛行特性}
\label{flight_char}
%%%%%%%%%%%%%%%%%%%%%%

本章では,実験で得られたデータから,これまでに述べた力学モデルを用いてパラメータ同定を行ない,低速飛行時の飛行特性について検証する.まず機体が持つ固有振動をとらえるために,線形モデルによる固有値解析を行なう.次に,同定を行なった結果から,設定した空気力モデルの妥当性を検討する.最後に,同定結果をCFD解析の結果と比較し,考察を行なう.

%%%%%%%%%%%%%%%%%%%%%%%%%%%%%%%%%%%%
\section{線形モデルによる固有値解析}
\label{sec:analyze}
%%%%%%%%%%%%%%%%%%%%%%%%%%%%%%%%%%%%

本節では,線形化された機体の飛行モデルを用いて,固有値解析を行なう.まず,式(\ref{eq:lin_model})より,$\Delta$を省略して微分方程式をまとめると
\begin{equation}
  \dfrac{d}{dt}
  \underbrace{
  \left[
  \begin{array}{cccc}
    u \\
    \alpha \\
    q \\
    \theta \\
  \end{array}
  \right]}_{\underline{x}} =
  \underbrace{
  \left[
  \begin{array}{cccc}
    X_u & X_\alpha & X_q & X_\theta \\
    \overline{Z_u} & \overline{Z_\alpha} & \overline{Z_q} & \overline{Z_\theta} \\
    \overline{M_u} & \overline{M_\alpha} & \overline{M_q} & \overline{M_\theta} \\
    0 & 0 & 1 & 0
  \end{array}
  \right]}_{A}
  \underbrace{
  \left[
  \begin{array}{cccc}
    u \\
    \alpha \\
    q \\
    \theta \\
  \end{array}
  \right]}_{\underline{x}} +
  \underbrace{
  \left[
  \begin{array}{cccc}
    X_{\delta_e} & X_{T_m} & 0 & 0 \\
    \overline{Z_{\delta_e}} & \overline{Z_{T_m}} & \overline{Z_{T_r}} & \overline{Z_{T_f}} \\
    \overline{M_{\delta_e}} & \overline{M_{T_m}} & \overline{M_{T_r}} & \overline{M_{T_f}} \\
    0 & 0 & 0 & 0
  \end{array}
  \right]
  \left[
  \begin{array}{cccc}
    \delta_e \\
    T_m \\
    T_r \\
    T_f \\
  \end{array}
  \right]}_{\underline{u}}
\end{equation}
となる.プロセスノイズを省略すれば,\cite{}を参考に,この状態方程式は行列$A$と状態量$\underline{x}(u,\alpha,q,\theta)$,入力$\underline{u}(\delta_e,T_m,T_r,T_f)$を用いることによって
\begin{equation}
  \dfrac{d\underline{x}}{dt} = A\underline{x} + \underline{u}
\label{eq:matrix_A}
\end{equation}
と表すことができる.ここで入力$\underline{u}$が,振動や減衰を表す関数
\begin{equation}
  \underline{u} = \sum \underline{f} e^{\omega t}
\end{equation}
であるとする.ただし$\omega$は複素数($\omega \in \mathbb{C}$)である.これにより,状態量$\underline{x}$は解析的に解くことができ
\begin{equation}
  \underline{x} = \left(\sum_{i}C_i x_i e^{\lambda_i t}\right) +
  \sum(\omega I - A)^{-1} \underline{f} e^{\omega t}
\end{equation}
となる.ここで右辺第1項は,ある係数$C_i$,行列$A$の固有値$\lambda_i$,固有ベクトル$x_i$で表された一般解で,状態に固有な振動や発散,減衰などを表している.右辺第2項は特殊解であり,入力と同じ周波数成分を持つ.つまり,外乱がない限り,状態量は周波数成分として固有振動数や入力に存在する周波数成分に相関するということである.

そこで,\ref{sec:filter}でも述べたように,固有振動数を計算することで機体の運動が持つおおよその周波数帯をつかみ,データのフィルタリングに利用する.実際に式(\ref{eq:matrix_A})の行列$A$について,固有値を計算してプロットしたものがFig. \ref{fig:}である.

%%%%%%%%%%%%%%%%%%%%%%%%%%%%
\section{空気力モデルの検証}
%%%%%%%%%%%%%%%%%%%%%%%%%%%%

%%%%%%%%%%%%%%%%%%%%%%%%%%%%%%%
\section{CFDによる解析結果との比較}
%%%%%%%%%%%%%%%%%%%%%%%%%%%%%%%
