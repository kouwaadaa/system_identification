% 第5章 低速飛行特性
%!TEX root = main.tex

%%%%%%%%%%%%%%%%%%%%%%
\chapter{低速飛行特性}
\label{flight_char}
%%%%%%%%%%%%%%%%%%%%%%

本章では,実験で得られたデータから,これまでに述べた力学モデルを用いてパラメータ同定を行ない,低速飛行時の飛行特性について検証する.まず機体が持つ固有振動をとらえるために,線形モデルによる固有値解析を行なう.次に,同定を行なった結果から,設定した空気力モデルの妥当性を検討する.最後に,同定結果をCFD解析の結果と比較し,考察を行なう.

%%%%%%%%%%%%%%%%%%%%%%%%%%%%%%%%%%%%
\section{線形モデルによる固有値解析}
\label{sec:analyze}
%%%%%%%%%%%%%%%%%%%%%%%%%%%%%%%%%%%%

本節では,線形化された機体の飛行モデルを用いて,固有値解析を行なう.まず,式(\ref{eq:lin_model})より,$\Delta$を省略して微分方程式をまとめると
\begin{equation}
  \dfrac{d}{dt}
  \underbrace{
  \left[
  \begin{array}{cccc}
    u \\
    \alpha \\
    q \\
    \theta \\
  \end{array}
  \right]}_{\underline{x}} =
  \underbrace{
  \left[
  \begin{array}{cccc}
    X_u & X_\alpha & X_q & X_\theta \\
    \overline{Z_u} & \overline{Z_\alpha} & \overline{Z_q} & \overline{Z_\theta} \\
    \overline{M_u} & \overline{M_\alpha} & \overline{M_q} & \overline{M_\theta} \\
    0 & 0 & 1 & 0
  \end{array}
  \right]}_{A}
  \underbrace{
  \left[
  \begin{array}{cccc}
    u \\
    \alpha \\
    q \\
    \theta \\
  \end{array}
  \right]}_{\underline{x}} +
  \underbrace{
  \left[
  \begin{array}{cccc}
    X_{\delta_e} & X_{T_m} & 0 & 0 \\
    \overline{Z_{\delta_e}} & \overline{Z_{T_m}} & \overline{Z_{T_r}} & \overline{Z_{T_f}} \\
    \overline{M_{\delta_e}} & \overline{M_{T_m}} & \overline{M_{T_r}} & \overline{M_{T_f}} \\
    0 & 0 & 0 & 0
  \end{array}
  \right]}_{B}
  \underbrace{
  \left[
  \begin{array}{cccc}
    \delta_e \\
    T_m \\
    T_r \\
    T_f \\
  \end{array}
  \right]}_{\underline{u}}
\end{equation}
となる.プロセスノイズを省略すれば,この状態方程式は行列$A,B$と状態量$\underline{x}(u,\alpha,q,\theta)$,入力$\underline{u}(\delta_e,T_m,T_r,T_f)$を用いることによって
\begin{equation}
  \dfrac{d\underline{x}}{dt} = A\underline{x} + B\underline{u}
\label{eq:matrix_A}
\end{equation}
と表すことができる.したがってこれを連続時間システムとすれば,一般に
\begin{equation}
  \underline{x}(t) = e^{At}\underline{x}(0) + \int_0^t e^{A(t-\tau)}B\underline{u}(\tau) d\tau
  \label{eq:system}
\end{equation}
が得られる.式(\ref{eq:system})の右辺第1項目は,斉次方程式の一般解であり,行列$A$の固有値から成る.右辺第2項目は特殊解であり,\underline{u}による振動数と行列$A$の固有値から成る.例えば,右辺第2項目は$u=1$のとき,$\int_0^t e^{A(t-\tau)}B d\tau$となり,行列$A$の固有値に依存した振動になり得るということである.

% ここで入力$\underline{u}$が,振動や減衰を表す関数
% \begin{equation}
%   \underline{u} = \sum \underline{f} e^{\omega t}
% \end{equation}
% であるとする.ただし$\omega$は複素数($\omega \in \mathbb{C}$)である.これにより,状態量$\underline{x}$は解析的に解くことができ
% \begin{equation}
%   \underline{x} = \left(\sum_{i}C_i x_i e^{\lambda_i t}\right) +
%   \sum(\omega I - A)^{-1} \underline{f} e^{\omega t}
% \end{equation}
% となる.ここで右辺第1項は,ある係数$C_i$,行列$A$の固有値$\lambda_i$,固有ベクトル$x_i$で表された一般解で,状態に固有な振動や発散,減衰などを表している.右辺第2項は特殊解であり,入力と同じ周波数成分を持つ.つまり,外乱がない限り,状態量は周波数成分として固有振動数や入力に存在する周波数成分に相関するということである.


そこで,\ref{sec:filter}小節でも述べたように,固有振動数を計算することで機体の運動が持つおおよその周波数帯をつかみ,データのフィルタリングに利用する.実際に式(\ref{eq:matrix_A})の行列$A$について,固有値を計算して絶対値をプロットしたものがFig. \ref{fig:eigenvalue}である.ただし,実験中の風の影響などの外乱が大きいと思われるデータは除き,計算を行なった.行列$A$が4次の正方行列であるため,各データ点ごとに最大4つの固有値を持ち,それぞれ色分けされている.グラフ内の縦線は,複数の実験データそれぞれの境界を示す.

Fig. \ref{fig:eigenvalue}から,機体の回転翼機モードにおける低速飛行縦運動の固有振動数は,おおよそ0〜5$\mathrm{[Hz]}$であることが見て取れる.つまり,この範囲内の周波数はフィルタリングなどの処理で落としてはいけない.このように小型UAVにおいても,運動の固有値解析を行なうことは有用であると考えられる.

\begin{figure}[H]
	\centering
	\includegraphics[clip,width=15.0cm,bb=0 0 1250 750]{./z_figure_files/chapter5/1_eigenvalue.jpeg}
	\caption{Eigenvalue of matrix A}
	\label{fig:eigenvalue}
\end{figure}

%%%%%%%%%%%%%%%%%%%%%%%%%%%%
\section{空気力モデルの検証}
\label{sec:airf_model_ver}
%%%%%%%%%%%%%%%%%%%%%%%%%%%%

本節では,パラメータ同定の結果から,式(\ref{eq:L})〜(\ref{eq:Ma})で設定した空気力モデルが妥当であるかを考える.Fig. \ref{fig:CL_si}〜\ref{fig:Cm_si}に示すように,$L_{log},D_{log},M_{a_{log}}$から算出した空力係数(青点),空気力モデル内の速度に比例した項を除いたモデルを用いて同定した結果から算出した空力係数(黃点),および本研究で仮定する空気力モデルを用いて同定した結果から算出した空力係数(緑点)とをそれぞれ比較する.ただし,式(\ref{eq:CL}),~(\ref{eq:Cm})より空力係数に関係する変数は,$\alpha,\dot{\alpha},q,\delta_e,V_a$の計5個あるため,その中から$V_a$を横軸にとった射影を図示している.

$C_L$はそれほど大きな変化は見られないが,$C_D,C_m$では$k_* V_a$の項を含めた空気力モデルの方が,実験値に沿うように点在しているように見える.

\begin{figure}[htbp]
	\begin{center}
		\begin{tabular}{c}
			\begin{minipage}{0.5\hsize}
				\begin{center}
					\includegraphics[clip,width=7.5cm,bb=0 0 982 835]{./z_figure_files/chapter5/2_CL.jpeg}
					\caption{$C_L$ plots}
					\label{fig:CL_si}
				\end{center}
			\end{minipage}
			\begin{minipage}{0.5\hsize}
				\begin{center}
					\includegraphics[clip,width=7.5cm,bb=0 0 982 835]{./z_figure_files/chapter5/3_CD.jpeg}
					\caption{$C_D$ plots}
					\label{fig:CD_si}
				\end{center}
			\end{minipage}
		\end{tabular}
	\end{center}
\end{figure}
\begin{figure}[H]
  \begin{center}
    \includegraphics[clip,width=7.5cm,bb=0 0 982 835]{./z_figure_files/chapter5/4_Cm.jpeg}
    \caption{$C_m$ plots}
    \label{fig:Cm_si}
  \end{center}
\end{figure}


%%%%%%%%%%%%%%%%%%%%%%%%%%%%%%%
\section{CFDによる解析結果との比較}
\label{sec:cfd}
%%%%%%%%%%%%%%%%%%%%%%%%%%%%%%%

本節では,パラメータ同定結果の妥当性をさらに検証するために,CFDによる解析結果とも比較する.CFD解析において,パラメータ同定の計算上の条件と比べて,機体前方から一定風が吹いているという条件は一致するが,常に対気速度$V_a=5\mathrm{[m s^{-1}]}$であるという条件が異なる.そのため,同定結果およびログデータのうち,$V_a$が$4.5\mathrm{[m s^{-1}]}$以上$5.5\mathrm{[m s^{-1}]}$以下であるデータのみを採用した.Fig. \ref{fig:cfd_L}〜\ref{fig:cfd_Ma}に,迎角$\alpha$を横軸にとり,揚力・抗力・ピッチモーメントそれぞれの空力係数をプロットしたものを示す.青の点がログデータから算出された値,緑の点が同定結果から再現された値,そして赤の点がCFD解析による値である.また,$\alpha$との関係性を見るために,各点と同じ色で回帰直線を引いてある.

$C_L$について,CFDの解析結果は$\alpha$が大きいほど数値も大きくなることが見て取れる.しかし,ログデータや,特に同定結果を見ると,$\alpha$が$-5\mathrm{[deg]}$付近を境に$\alpha$の上昇に伴って数値が小さくなっていくように見える.CFDでは,ロータは回していない状態で解析を行なっているため,$\alpha$が大きいとき,ロータによって発生する空気の流れによって,負の揚力が発生しているのではないかと考えられる.

次に$C_D$について,同定結果とCFD解析結果がほぼ一致している一方で,ログデータとは一致していない.これは抗力係数のうち,迎角によって変化するとされる有害抵抗係数による影響だと考えられる\cite{}.同定に用いる式(\ref{eq:D_o})の設定を見直す必要があると思われる.

最後に$C_m$について,ログデータにはばらつきがあるものの,どの結果も$\alpha$の値によらずほぼ一定の値を取っている.CFDの解析結果が大きく離れているのは,ロータ推力の推算による誤差が大きい可能性が考えられる.ロータ推力の算出方法も改善の余地があると思われる.

\begin{figure}[htbp]
	\centering
		\begin{tabular}{c}
			\begin{minipage}{0.5\hsize}
				\centering
					\includegraphics[clip,width=7.5cm,bb=0 0 1912 1743]{./z_figure_files/chapter5/cfd_L.jpeg}
					\caption{Comparing $C_L$ results of CFD and identification}
					\label{fig:cfd_L}
			\end{minipage}
			\begin{minipage}{0.5\hsize}
				\centering
					\includegraphics[clip,width=7.5cm,bb=0 0 1819 1662]{./z_figure_files/chapter5/cfd_D.jpeg}
					\caption{Comparing $C_D$ results of CFD and identification}
					\label{fig:cfd_D}
			\end{minipage}
		\end{tabular}
\end{figure}
\begin{figure}[H]
  \centering
    \includegraphics[clip,width=7.5cm,bb=0 0 1832 1662]{./z_figure_files/chapter5/cfd_m.jpeg}
    \caption{Comparing $C_m$ results of CFD and identification}
    \label{fig:cfd_Ma}
\end{figure}
