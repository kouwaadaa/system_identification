% 第1章
電力の消費は,季節によっても,また,1日の中でも昼間と夜間では大きな差がある.
このような電力需要の変化に対応するために,様々な電源をバランスよく組み合わせて%
発電を行なうことが必要となる.
水力発電は他の電源と比較して,短い時間で発電開始が可能であり,%
また,発電機出力の安定性や負荷変動に対する追従性の面で優れており,%
1日の電力需要のピークを支える役割が期待される.
その一方で,水力発電は上流の貯水施設でせき止めた水を放流することによって行なわれるため,%
周辺の自然環境や生活環境に及ぼす影響にも配慮する必要がある.
そのため,環境の維持などの条件を満たしつつ,%
貯水施設や発電機から成る水系を効率良く運用する計画の立案が肝要である.\\
\indent
水系運用に関する研究・開発についてみると,水系運用を支援する方策・システムに関するものが数多く報告されているものの\cite{develop,method,alg} ,%
水系運用計画を取り扱ったものはほとんどない\cite{enhancement,dp}.
また,現状の水系運用計画・支援システムでは種々の制約の充足が不十分であり,%
運用(計画)の良し悪しは現場の人手に委ねられる部分が多く,計画の作成に時間を要する上に,%
必ずしも水資源が効率良く運用されているとはいえないという問題点がある.\\
\indent
そこで,本研究では水資源の効率的な運用を目的とし,必要となる制約をすべて盛り込んだ%
数理計画モデルを構築する.
さらに,構築した数理計画モデルを例題に適用した結果を示し,%
構築したモデルの妥当性について検討する.\\
\indent
本論文の構成は以下の通りである.まず,第2章では本研究で対象とする水系運用計画最適化問題の一般的な定義を行なう.第3章では,第2章で示した定義に基づいて,水系運用計画最適化問題の%
数理計画モデル化を行なう.第4章では計算例を通して,第3章で記述したモデルの妥当性を%
検討する.第5章では,結論として本研究で得られた成果と今後の課題を要約する.