\begin{titlepage}
%%% ページ番号つける?そのままだと番号がつきます.
%\thispagestyle{empty}
\begin{center}
{\LARGE \bf ティルトロータ型UAVにおける\\低速飛行特性の解析}
\\[0.5cm]
{\Large 弓場~洋輝}
\\[1.0cm]
{\LARGE \bf 要\vspace{36pt}   旨\\}
\end{center}

\indent

貯水施設や発電機がある放流路など様々な放流路から構成される水系の運用について,%
水系に存在する種々の制約を満たしつつも,効率的に水資源を利用する水系運用計画を%
作成することが肝要である.本研究では,効率的かつ実用的な水系運用計画の作成問題を%
対象として,数理計画モデルの一構成法を示すとともに,その妥当性について検討する.
\\
\indent
現状では,水系支援システムは種々の制約の充足などの機能が十分ではなく,%
実用的なものではない.そのため,運用計画の作成は%
現場の人手による対応に委ねられる部分が多くなっている.
本研究では,まず水系運用最適化問題を一般的に定義し,その定義に基づき,%
数理計画モデルとして定式化を行なう.
構成した数理計画モデルを現実的な水系を対象とした計算例を通して,モデルの妥当性について%
検討する.
\\
\indent
計算例より,制約を満たしつつ,水の効率的な運用がなされていることが確認できた.
この結果より提案モデルが一定の妥当性を有するものと考えられる.
水系運用計画を現実の運用に耐え得るものとするための,モデルのさらなる拡充・改良が今後の課題となる.

\end{titlepage}
