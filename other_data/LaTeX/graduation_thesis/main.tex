\documentclass[12pt,epsf,here,fleqn]{jreport}
\usepackage[dvipdfmx]{graphicx}
\usepackage[dvipdfmx]{color}
\usepackage{here}
\usepackage{fancybox}
\usepackage{psfrag}
\usepackage{amsmath,amssymb}
\usepackage{mediabb}

\def\figurename{Figure}
\def\tablename{Table}
\def\bibname{参考文献}

\setlength{\topmargin}{-1.0cm}
\setlength{\oddsidemargin}{1.0cm}
\setlength{\evensidemargin}{1.0mm}
\setlength{\textwidth}{15cm}
\setlength{\textheight}{23cm}
\setcounter{topnumber}{5}%    ページ上部の図表は 5 個まで
\def\topfraction{1.00}%       ページの上 1.00 まで図表で占めて可
\setcounter{bottomnumber}{5}% ページ下部の図表は 5 個まで
\def\bottomfraction{1.00}%    ページの下 1.00 まで図表で占めて可
\setcounter{totalnumber}{10}% ページあたりの図表は 10 個まで
\def\textfraction{0.04}%      ページうち本文が占める割合の下限
%        これを 0 にすると本文が 1 行だけのページが出来る
%        0.04 くらいにすると 1 行だけのページは防げる
%        0.1 くらいが良いかも知れない
\def\floatpagefraction{0.7}%  図表だけのページは少なくとも
                           %  これだけを図表が占める

\begin{document}%%%%%%%%ドキュメントの始まり
\baselineskip=0.8cm%%%%%タイトルの行間隔
%\documentclass[report, a4paper,11pt]{../thesis}
\usepackage[dvips]{graphicx}
\usepackage{ascmac} %枠で囲むやつら

%\renewcommand{\thesection}{\arabic{chapter}.\arabic{section}}
%\renewcommand{\thesubsection}{\arabic{chapter}.\arabic{section}.\arabic{subsection}}

\makeatletter       %参考文献を[1]から(1)へさらにciteを上付きへ
  \renewcommand{\@biblabel}[1]{(#1)}
  \DeclareRobustCommand\cite{\unskip
  \@ifnextchar[{\@tempswatrue\@citex}{\@tempswafalse\@citex[]}}
  \def\@cite#1#2{$\!^{\hbox{\scriptsize{(#1\if@tempswa , #2\fi})}}$}
  \def\@biblabel#1{#1)}
  \@addtoreset{figure}{chapter}
\makeatother

\addtolength{\oddsidemargin}{31truemm}
\addtolength{\textwidth}{-55truemm}
\setlength{\fullwidth}{\textwidth} 
%\addtolength{\topmargin}{-1truein}
\setlength{\headheight}{0pt}
\setlength{\headsep}{0pt}
\addtolength{\textheight}{20truemm}
\renewcommand{\baselinestretch}{1.2}
\newcommand{\figref}[1]{Fig:\ref{#1}}
\newcommand{\tableref}[1]{Table \ref{#1}}

\begin{document}

\pagestyle{empty}
%-----------------------------
%表紙
%-----------------------------
\newpage
\begin{center}
\vspace*{1cm}
{\Huge{○○年度}}\\
\vspace*{0.5cm}
{\Huge{卒業論文}}\\
\vspace*{2cm}
{\LARGE {○○○○○○の○○○○○○における\\○○○○○○の研究}}\\
\vspace*{5cm}
{\LARGE{神戸大学工学部情報知能工学科}}\\
\vspace*{1.0cm}
{\LARGE{神戸 太郎}}\\
\vspace*{1.5cm}
{\LARGE{\underline{指導教員\hspace{1.0cm}○○ ○○教授}}}\\
\vspace{3.0cm}
{\LARGE{2007年2月26日}}
\end{center}
\input{../end}%%%%%%外タイトル
\newcommand{\hdate}[1]{\setcounter{page}{0}\begin{center}{\LARGE
#1\\}\vskip 20pt {\Huge 卒 業 論 文}\end{center}\vskip 40pt}

%題目が入ります
\newcommand{\htitle}[1]{
        \begin{center}
%                {\LARGE 題目}\\
                \vskip 16pt
                {\LARGE #1}\\
                \setlength{\unitlength}{1mm}
%               \vskip -20pt
%               \begin{picture}(100,10)(0,0)
%                       \put(0,0){\thicklines\line(1,0){100}}
%               \end{picture}
        \end{center}
        \vskip 40pt
}

\begin{titlepage}
\vspace*{2cm}
\hdate{2018年度}
\htitle{ティルトロータ型UAVにおける\\低速飛行特性の解析}

\hspace{8cm}
\begin{center}
\Large
神戸大学工学部情報知能工学科\\
\vskip 20pt
弓場~洋輝\\
\vskip 40pt
\underline{指導教員~~~~{\Large 浦久保~孝光~~准教授}}\\
%主査~~~~{\Large 玉置~久}\\
%副査~~~~{\Large 太田~能}\\
\vskip 20pt
%提出\\
2019年2月14日
\end{center}
\end{titlepage}
%%%%%%%内タイトル
\baselineskip=0.9cm%%%%%要旨の行間隔

\vspace*{13cm}
\begin{flushleft}
\hspace*{-9mm}
\includegraphics[height=8.3cm,width=6.0cm,clip]{z_figure_files/author.pdf}
\end{flushleft}
\vspace*{-5mm}
\hspace*{-9mm}
Copyright \copyright \, 2019, Hiroki Yuba


%%B4の人は省く----------------------------------------------------
%\thispagestyle{empty}%%%英文要旨ページ番号なし
%\begin{center}
{\LARGE \bf 英語のタイトル}
\\[0.5cm]
{\Large 英語で自分の名前}
\\[1.0cm]
{\LARGE \bf ABSTRACT}
\end{center}


\indent

English ABST を書く.
%%%%%%%英文要旨
%%----------------------------------------------------------------
\thispagestyle{empty}%%%和文要旨ページ番号なし
% 要旨
%!TEX root = main.tex

\begin{titlepage}
%%% ページ番号つける?そのままだと番号がつきます.
%\thispagestyle{empty}
\begin{center}
{\LARGE \bf ティルトロータ型UAVにおける\\低速飛行特性の解析}
\\[0.5cm]
{\Large 弓場~洋輝}
\\[1.0cm]
{\LARGE \bf 要\vspace{36pt}   旨\\}
\end{center}


\end{titlepage}
%%%%%%%和文要旨
%\baselineskip=0.7cm%%%%目次の間隔狭い時ON
\baselineskip=0.9cm%%%%%目次の行間隔
\pagenumbering{roman}%%%目次のページ番号roman
\tableofcontents%%%%%%%%目次
\newpage%%%%%%%%%%%%%%%%改ページ
\baselineskip=0.7cm%%%%%本文の行間隔
\pagenumbering{arabic}%%本文のページ番号arabic
\setcounter{page}{1}%%%%本文のページ1から開始

\chapter{序論}
  % 第1章
電力の消費は,季節によっても,また,1日の中でも昼間と夜間では大きな差がある.
このような電力需要の変化に対応するために,様々な電源をバランスよく組み合わせて%
発電を行なうことが必要となる.
水力発電は他の電源と比較して,短い時間で発電開始が可能であり,%
また,発電機出力の安定性や負荷変動に対する追従性の面で優れており,%
1日の電力需要のピークを支える役割が期待される.
その一方で,水力発電は上流の貯水施設でせき止めた水を放流することによって行なわれるため,%
周辺の自然環境や生活環境に及ぼす影響にも配慮する必要がある.
そのため,環境の維持などの条件を満たしつつ,%
貯水施設や発電機から成る水系を効率良く運用する計画の立案が肝要である.\\
\indent
水系運用に関する研究・開発についてみると,水系運用を支援する方策・システムに関するものが数多く報告されているものの\cite{develop,method,alg} ,%
水系運用計画を取り扱ったものはほとんどない\cite{enhancement,dp}.
また,現状の水系運用計画・支援システムでは種々の制約の充足が不十分であり,%
運用(計画)の良し悪しは現場の人手に委ねられる部分が多く,計画の作成に時間を要する上に,%
必ずしも水資源が効率良く運用されているとはいえないという問題点がある.\\
\indent
そこで,本研究では水資源の効率的な運用を目的とし,必要となる制約をすべて盛り込んだ%
数理計画モデルを構築する.
さらに,構築した数理計画モデルを例題に適用した結果を示し,%
構築したモデルの妥当性について検討する.\\
\indent
本論文の構成は以下の通りである.まず,第2章では本研究で対象とする水系運用計画最適化問題の一般的な定義を行なう.第3章では,第2章で示した定義に基づいて,水系運用計画最適化問題の%
数理計画モデル化を行なう.第4章では計算例を通して,第3章で記述したモデルの妥当性を%
検討する.第5章では,結論として本研究で得られた成果と今後の課題を要約する.
\chapter{実験機体}
  \label{aircraft}
  % 第2章
%!TEX root = main.tex

%%%%%%%%%%%%%%%%%%
\chapter{実験機体}
\label{aircraft}
%%%%%%%%%%%%%%%%%%

本章では,実験を行なった開発機体の開発コンセプトと,設計や搭載システムの概要を述べる.機体は,災害発生時,回転翼機モードで離陸し,上空で固定翼機モードへと遷移して被災地へ向かう.そして被災地上空へ到着した後,回転翼機モードへと遷移し,ホバリング飛行しながら,情報収集や着陸可能地点の検出を行なう.

\section{実験機の概要}
本研究グループの目的である,大規模災害発生時の任務遂行には,狭隘地への進入が必要な場合がある.また救援物資の運搬に利用する場合,救助者に近い距離で着陸を行なう可能性もあり,対人安全性の強化が必要である.さらに,空撮や着陸可能地点の検出には,安定した飛行とホバリングを行なう必要がある.

以上を踏まえて,機体製作にあたり以下の3点
	\begin{enumerate}
	\item 機体サイズの小型化
	\item 対人安全性を考慮
	\item ヘリコプタと同等のホバリング性能
	\end{enumerate}
をコンセプトとしている.2015年に,エアロセンス株式会社と共同開発を行なったが,本研究ではFig. \ref{fig:vtol23k}に示す後継機を用いている.

\hspace{5pt}

メインロータには,同軸二重反転ロータを使用する.二重反転ロータはシングルロータと比較して効率がよく,推力が大きいと言われている.また,ヘリコプタモード時に反トルクを打ち消すためのテールロータが不要になり,機体サイズの小型化を図れるという利点もある.この二重反転ロータユニットを胴体中央部に組み込むことで,ロータの機体外部への露出を防止し,対人安全性の強化を図っている.

\begin{figure}[H]
	\centering
	\includegraphics[clip,width=10.0cm,bb=0 0 1478 1108]{./z_figure_files/chapter2/1_vtol23K.JPG}
	\caption{Tilt rotor UAV}
	\label{fig:vtol23k}
\end{figure}

上下に取り付けられたメインロータは,それぞれ独立に駆動可能であるため,上下のロータで異なる回転数を実現できる.ホバリング時のヨー制御は,上下のロータの回転数を調整することで発生する反トルクの差を利用して行なう.メインロータを取り付けたティルト軸はサーボ機構を有しており,機体胴体に対してピッチ軸周りの回転自由度を有する.Fig. \ref{fig:tilt0} ~ \ref{fig:tilt90}に示すように,メインロータをティルトさせることで,ヘリコプタモード,遷移モード,飛行機モードの切り替えを行う.
\begin{figure}[h]
	\begin{center}
		\begin{tabular}{c}

			\begin{minipage}{0.33\hsize}
				\begin{center}
					\includegraphics[clip,width=4.0cm,bb=0 0 4032 3024]{./z_figure_files/chapter2/2_tilt0.JPG}
					\caption{Helicopter Mode}
					\label{fig:tilt0}
				\end{center}
			\end{minipage}

			\begin{minipage}{0.33\hsize}
				\begin{center}
					\includegraphics[clip,width=4.0cm,bb=0 0 4032 3024]{./z_figure_files/chapter2/3_tilt45.JPG}
					\caption{Transition Mode}
					\label{fig:tilt45}
				\end{center}
			\end{minipage}

			\begin{minipage}{0.33\hsize}
				\begin{center}
					\includegraphics[clip,width=4.0cm,bb=0 0 4032 3024]{./z_figure_files/chapter2/4_tilt90.JPG}
					\caption{Airplane Mode}
					\label{fig:tilt90}
				\end{center}
			\end{minipage}

		\end{tabular}
	\end{center}
\end{figure}

 ヘリコプタモードにおいて,風などの外乱に対して安定したホバリングを行なうために,姿勢制御用のサブロータを左右翼端に1つずつ,機首部上下に2つ,計4つのサブロータを取り付けている(Fig. \ref{fig:sub1},~\ref{fig:sub2}).機首部のサブロータは二重反転ロータになっており,機首部のサブロータで発生する反トルクを打ち消すようになっている.ヘリコプタモードにおけるホバリング時には,サブロータとメインロータの出力やティルト角を調整し,機体重心周りのモーメントを発生させることでピッチ軸およびロール軸の制御を行なう.

\begin{figure}[h]
 \begin{center}
	 \begin{tabular}{c}

		 \begin{minipage}{0.49\hsize}
			 \begin{center}
				 \includegraphics[clip,width=6.0cm,bb=0 0 1920 1080]{./z_figure_files/chapter2/5_Sub_rotor_wing.JPG}
				 \caption{Sub rotor (wing tip)}
				 \label{fig:sub1}
			 \end{center}
		 \end{minipage}

		 \begin{minipage}{0.49\hsize}
			 \begin{center}
				 \includegraphics[clip,width=6.0cm,bb=0 0 4096 2304]{./z_figure_files/chapter2/6_Sub_rotor_nose.JPG}
				 \caption{Sub rotor (nose)}
				 \label{fig:sub2}
			 \end{center}
		 \end{minipage}

	 \end{tabular}
 \end{center}
\end{figure}

また,姿勢制御のための動翼として,左右で独立に動かすことができるエレボン(Fig. \ref{fig:elevon})が主翼後縁に取り付けられており,エルロンとしてのロール角制御と,エレベータとしてのピッチ角制御を行うことができる.また,垂直尾翼後縁には,ヨー角制御に用いるラダー(Fig. \ref{fig:rudder})が取り付けられている.\\

\begin{figure}[hb]
	\begin{center}
		\begin{tabular}{b}

			\begin{minipage}{0.49\hsize}
				\begin{center}
					\includegraphics[clip,width=6.0cm,bb=0 0 1067 800]{./z_figure_files/chapter2/7_elevon.jpg}
					\caption{Elevon}
					\label{fig:elevon}
				\end{center}
			\end{minipage}

			\begin{minipage}{0.49\hsize}
				\begin{center}
					\includegraphics[clip,width=6.0cm,bb=0 0 1067 800]{./z_figure_files/chapter2/8_rudder.jpg}
					\caption{Rudder}
					\label{fig:rudder}
				\end{center}
			\end{minipage}

		\end{tabular}
	\end{center}
\end{figure}

Fig. \ref{fig:frame}は機体フレームである.フレームにはカーボンパイプを使用し,機体の軽量化を図りつつ剛性も確保している.また,Table \ref{spec}に機体サイズや機体を構成するハードウェア機器に関してまとめる.最後に,Fig. \ref{fig:wings},Table \ref{wingspec}に主翼と尾翼の面積や長さについてまとめる.面積や長さには,エレボンやラダーも含まれている.

%%%%%%%%%%%%%%%%%%%%%%
\section{搭載システム}
%%%%%%%%%%%%%%%%%%%%%%

正確な機体の運動制御のためには,機体の速度や姿勢などの飛行データを正確に得ることが必要である.本研究では,3軸加速度・角速度センサ,地磁気センサ,気圧センサを内蔵した小型の搭載コンピュータ(Fig. \ref{fig:pixhawk}),GPSセンサ(Fig. \ref{fig:GPS}),およびピトー管とよばれる対気速度計測装置(Fig. \ref{fig:pitot})を使用する.これらのセンサからの観測値をもとに,カルマンフィルタによって飛行状態が推定される.また,搭載コンピュータを用いて各ロータ推力などの制御量の計算を行なう.
\begin{table}[htb]
	\begin{center}
		\caption{Specifications of the tilt rotor UAV}
		\label{spec}
		\begin{tabular}{|c|c|}\hline
			Total length & 1.08[m] \\ \hline
			Total width & 1.43[m] \\ \hline
			Total weight & 5.738[kg]\\ \hline
			Main rotor motor & KDE4215XF-435 $\times$ 2\\ \hline
			Main propeller(upper) & MB-ARRIS-P1365C$13 \times 6.5$EP\\ \hline
			Main propeller(lower) & APC$13 \times 8.0$E\\ \hline
			Main rotor battery & 6S-6300mAh(30C)\\ \hline
			Main rotor maximum thrust(coaxial) & 6.209[kgf]\\ \hline
			Sub rotor motor & RM2204-2600 $\times$ 4\\ \hline
			Sub rotor propeller & Lumenier $4 \times 4 \times 4$\\ \hline
			Sub rotor and System battery & 4S-2200mAh(35C)\\ \hline
			Sub rotor maximum thrust(single) & 0.931[kgf]\\ \hline
		\end{tabular}
	\end{center}
\end{table}
\begin{figure}[hb]
	\begin{center}
		\includegraphics[clip,width=10.0cm,bb=0 0 1122 543]{./z_figure_files/chapter2/9_frame.jpeg}
		\caption{Body frame}
		\label{fig:frame}
	\end{center}
\end{figure}

\begin{figure}[h]
	\begin{center}
		\includegraphics[clip,width = 10.0cm, bb=0 0 1000 511]{./z_figure_files/chapter2/10_Wings.jpeg}
		\caption{Main wing and vertical wing}
		\label{fig:wings}
	\end{center}
\end{figure}

\begin{table}[h]
	\begin{center}
		\caption{Specifications of the wings}
		\label{wingspec}
		\begin{tabular}{|c|c|}\hline
			Wing area($S_w$) & $0.1677\times2$[m$^2$]\\ \hline
			Wing area($S_t$) & $0.0474\times2$[m$^2$]\\ \hline
			Length($l_1$) & $0.52$[m]\\ \hline
			Length($l_2$) & $0.57$[m]\\ \hline
			Length($l_3$) & $0.36$[m]\\ \hline
			Length($l_4$) & $0.23$[m]\\ \hline
		\end{tabular}
	\end{center}
\end{table}
\begin{table}[h]
	\begin{center}
		\caption{Sensors and onboard computer}
		\label{tab:Tab2.2}
		\begin{tabular}{|c|c|} \hline
			Onboard computer & 3D Robotics Pixhawk(CPU+IO)\\ \hline
			Acceleration and magnetic filed sensor & STMicroelectoronics LSM303D\\ \hline
			Gyro sensor & STMicroelectoronics L3GD20H\\ \hline
			Barometric pressure sensor & Measurement Specialties MS5611\\ \hline
			GPS sensor & u-blox LEA-6H\\ \hline
			Airspeed senser & MS4525DO\\ \hline
		\end{tabular}
	\end{center}
\end{table}
\begin{figure}[h]
	\begin{center}
		\begin{tabular}{c}

			\begin{minipage}{0.33\hsize}
				\begin{center}
					\includegraphics[clip,width=4.5cm,bb=0 0 1000 1333]{./z_figure_files/chapter2/11_pixhawk.JPG}
					\caption{Onboard computer with sensors}
					\label{fig:pixhawk}
				\end{center}
			\end{minipage}

			\begin{minipage}{0.33\hsize}
				\begin{center}
					\includegraphics[clip,width=4.5cm,bb=0 0 1000 1333]{./z_figure_files/chapter2/12_GPS.JPG}
					\caption{GPS sensor}
					\label{fig:GPS}
				\end{center}
			\end{minipage}

			\begin{minipage}{0.33\hsize}
				\begin{center}
					\includegraphics[clip,width=4.0cm,bb=0 0 500 654]{./z_figure_files/chapter2/13_pitot.png}
					\caption{Airspeed sensor}
					\label{fig:pitot}
				\end{center}
			\end{minipage}

		\end{tabular}
	\end{center}
\end{figure}

\chapter{力学モデル}
  \label{model}
  % 第3章

\section{座標系}

\section{縦運動の非線形モデル}

\section{空気力モデル}

\chapter{パラメータ同定}
  \label{sys_id}
  % 第4章

\section{データの前処理}

\section{パラメータの推定手法}

\chapter{低速飛行特性}
  \label{flight_char}
  % 第5章 低速飛行特性
%!TEX root = main.tex

\section{線形モデル}

\section{空気力モデルの検証}

\section{CFDの解析結果との比較}

\chapter{結論}
  % 5章
本研究では,


\newpage
\addcontentsline{toc}{chapter}{謝辞}
% 謝辞
%!TEX root = main.tex

\chapter*{謝辞}

本研究を進めるにあたり,丁寧なご指導を賜りました神戸大学大学院~システム情報学研究科~准教授~浦久保孝光先生に深く感謝の意を表します.また,非常に有益な御助言を頂戴致しました同大学院工学研究科~教授~玉置久先生に感謝致します.航空機力学という未知の分野ではありましたが,研究を通して様々な基礎分野を勉強させていただき,実験的に応用するところまで経験させていただきました.研究がいかに有意義なもので,勉学がいかに楽しいものであるか,短い期間ではありましたが,再確認させていただきました.熱心なご指導を頂いたことを重ねて感謝致します.

また共同研究として,佐部浩太郎様,平井真二様,村越象様をはじめとしたエアロセンス株式会社の皆様には,機体の開発にご協力いただきました.特に名前を挙げさせていただきました3名の方々には,実験の際に直接ご助言を頂いたこと,感謝致します.また,双葉電子工業~恩田哲男様,徳島大学~准教授~三輪昌史先生には,データ取得のための飛行実験の際に,パイロットとしてご協力いただきました.本当にありがとうございました.

そして,研究だけでなく,日々の生活においてもお世話になった同研究室の皆様にも深く感謝致します.加えて,私を様々な面で支えてくれた友人の皆様,そして何よりも私の大学生活を支えてくださった両親には感謝してもしきれません.

私が大学生活で関わったすべての皆様に,ここで改めて深く感謝申し上げます.

\newpage
%%%%謝辞
\addcontentsline{toc}{chapter}{参考文献}
% 参考文献
%!TEX root = main.tex

\begin{thebibliography}{99}

%\bibitem{ラベル} 著者: タイトル, 編集社, ページ (年)

\bibitem{katou}
加藤寛一郎 : ``航空機力学入門,'' , no. , pp.  ()

\end{thebibliography}

\newpage
%%%参考文献
\addtocontents{toc}{\protect\contentsline {chapter}{付録}{}}
\appendix
% \pagestyle{empty}
% \textheight=24cm
% \addtocounter{chapter}{1}
% \noindent
% {\Large \bf 付録}\\
%
% \noindent
%
% 付録です.
%%%%付録
\end{document}%%%%%%%%%%%ドキュメントの終了
