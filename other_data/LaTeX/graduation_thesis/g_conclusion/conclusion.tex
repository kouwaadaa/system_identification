% 結論
%!TEX root = main.tex

%%%%%%%%%%%%%%
\chapter{結論}
\label{conclusion}
%%%%%%%%%%%%%%

本論文では,開発機体である小型UAVの回転翼機モードを対象とし,パラメータ同定による飛行特性の取得と,飛行特性の解析について述べてきた.結論として,まずは各章ごとに概略をまとめる.
序論では小型UAVが大規模災害発生時に有用であることを示した上で,その中でも固定翼機と回転翼機の両方の長所を兼ね備えたティルトロータ型UAVに着目していることを述べた.加えて,我々が目指す高度な自律飛行のために,飛行特性の取得が重要であること,飛行特性の取得には一般の航空機に用いられているパラメータ同定手法を採用したことなどを述べた.研究対象を明確にするために,第2章では実験に使用した機体について述べた.

第3章では,開発機体の力学モデルを取り扱った.機体を単一の剛体とみなし,6自由度の非線形微分方程式および姿勢角の微分方程式を記述した.さらにそれらを縦方向と横方向の運動に分けて,本研究では縦運動にのみ着目し,飛行モデルを設定した.また,機体の微小擾乱を仮定することで飛行モデルの線形化を行なった.一般の航空機とは異なる空気力モデルの設定も同章で行なった.

第4章では,実機実験で得られた飛行ログデータを用いたパラメータ同定について述べた.まずデータの前処理として,直接得られないデータの算出方法や,フーリエ変換による周波数領域でのフィルタリング処理について説明した.また,最小二乗法による同定の手法を説明した後,実際の同定結果を示した.

第5章では,第4章で述べたパラメータ同定の結果もふまえ,回転翼機モードにおける低速飛行特性の解析およびモデルの妥当性の検証を行なった.また,CFDの解析結果との比較も行なっている.

\vspace{5pt}

本研究をふまえた上で提唱する今後の課題として,以下のような項目が挙げられる.
\begin{itemize}
  \item[(1)] より多く正確な実験データの取得
  \item[(2)] フィルタリング処理における入力項の考慮
  \item[(3)] パラメータ推定手法の検証
  \item[(4)] モデルの精度向上
\end{itemize}

以上に挙げた項目について一つずつ詳細を述べる.まず(1)について,\ref{eq:analyze}節で述べたように実験データの不足が課題とされる.$\dot{\alpha}$と$q$が独立になるように,例えば同じ姿勢を保ったまま前進かつ高度を変えるような飛行実験データを取得する必要がある.このように,より多くの条件での飛行データを,実験計画のもと取得することが課題であると思われる.また,実験中の風速の測定が正確に行えることも望ましい.

(2)について,(1)にも関連するが,より精度の高いフィルタリングを行なうために,式(\ref{eq:system})の入力項による影響をさらに考える必要がある.そのために,\cite{narioka}で用いられているウェーブレット解析による時間周波数情報の活用は有効であると考えられる.

(3)について,本研究ではパラメータ推定手法として最小二乗法を用いたが,最も基本的で容易な方法でありオフライン処理に限定された手法である.高度な自律飛行を実現するためにも,オンライン処理が可能であるほうが望ましいため,他の手法についても考慮する余地がある.

(4)について,本研究では小型UAVに関して,その形状と低速での飛行から一般的な航空機とは異なる空気力モデルを設定した.データ処理や推定の精度向上により,そのモデルの妥当性は本論でも述べてきたが,さらなるモデルの精度向上が期待できると考えている.
