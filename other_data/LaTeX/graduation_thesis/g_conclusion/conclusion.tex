% 結論
%!TEX root = main.tex

%%%%%%%%%%%%%%
\chapter{結論}
\label{conclusion}
%%%%%%%%%%%%%%

本論文では,開発機体である小型UAVの回転翼機モードを対象とし,パラメータ同定による飛行特性の取得と,飛行特性の解析について述べてきた.結論として,まずは各章ごとに概略をまとめる.
序論では小型UAVが大規模災害発生時に有用であることを示した上で,その中でも固定翼機と回転翼機の両方の長所を兼ね備えたティルトロータ型UAVに着目していることを述べた.加えて,我々が目指す高度な自律飛行のために,飛行特性の取得が重要であること,飛行特性の取得には一般の航空機に用いられているパラメータ同定手法を採用したことなどを述べた.研究対象を明確にするために,第2章では実験に使用した機体について述べた.

第3章では,開発機体の力学モデルを取り扱った.機体を単一の剛体とみなし,6自由度の非線形微分方程式および姿勢角の微分方程式を記述した.さらにそれらを縦方向と横方向の運動に分けて,本研究では縦運動にのみ着目し,飛行モデルを設定した.また,安定微係数による微小擾乱方程式を導出することで飛行モデルの線形化を行なった.一般の航空機とは異なる空気力モデルの設定もここで行なった.

第4章では,実機実験で得られた飛行ログデータを用いたパラメータ同定について述べた.まずデータの前処理として,直接得られないデータの算出方法や,フーリエ変換による周波数領域でのフィルタリング処理について説明した.また,最小二乗法による同定の手法を説明した後,実際の同定結果を示した.

第5章では,第4章で述べたパラメータ同定の結果もふまえ,回転翼機モードにおける低速飛行特性の解析およびモデルの妥当性の検証を行なった.また,CFDの解析結果との比較も行なっている.

\vspace{5pt}

本研究をふまえた上で提唱する今後の課題として,以下のような項目が挙げられる.
\begin{itemize}
  \item[(1)] より多く,より正確なデータの取得
  \item[(2)] 固有値解析における入力項の考慮
  \item[(3)] パラメータ推定手法の検証
  \item[(4)]
\end{itemize}

以上に挙げた項目について一つずつ詳細を述べる.(1)について,第4章でも述べた通り実験データには誤差が含まれている.これを少しでも正確なものにするために,データのフィルタリング処理の精度を高くする必要がある.また直接得られないため,ログデータから算出している値についても,算出方法を確立する必要があると思われる.

(2)について,(1)にも関連するが,より精度の高いフィルタリングを行なうために,式(\ref{eq:system})の入力項による影響をさらに考える必要がある.そのために,\cite{narioka}で用いられているウェーブレット解析による時間周波数情報の活用は有効であると考えられる.
