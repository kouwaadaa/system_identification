% 結論
%!TEX root = main.tex

%%%%%%%%%%%%%%
\chapter{結論}
\label{conclusion}
%%%%%%%%%%%%%%

本論文では,開発機体である小型UAVの回転翼機モードを対象とし,パラメータ同定による飛行特性の取得と,飛行特性の解析について述べてきた.結論として,まずは各章ごとに概略をまとめる.
序論では小型UAVが大規模災害発生時に有用であることを示した上で,その中でも固定翼機と回転翼機の両方の長所を兼ね備えたティルトロータ型UAVに着目していることを述べた.加えて,我々が目指す高度な自律飛行のために,飛行特性の取得が重要であること,飛行特性の取得には一般の航空機に用いられているパラメータ同定手法を採用したことなどを述べた.研究対象を明確にするために,第2章では実験に使用した機体について述べた.

第3章では開発機体の力学モデルを取り扱った.機体を単一の剛体とみなし,6自由度の非線形微分方程式および姿勢角の微分方程式を記述した.さらにそれらを縦方向と横方向の運動に分けて,本研究では縦運動にのみ着目し,飛行モデルを設定した.また,安定微係数による微小擾乱方程式を導出することで飛行モデルの線形化を行なった.
