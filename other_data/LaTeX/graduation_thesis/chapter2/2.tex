% 第2章
本章では,水系運用計画最適化問題について記述する.まず,水系運用計画最適化問題の概要に
ついて記述し,問題の基本要素と制約条件を述べる.

\section{問題の概要}
水系運用計画最適化問題とは,貯水施設と種々の放流路によって構成される水系において,%
各時間帯における放流路への放流量,特に,発電機が存在する放流路と貯水施設から水が溢れる%
こと(溢水)を防ぐために放流を行なう放流路への放流量を決定する問題である.
水系運用計画の作成には,貯水施設や各種放流路に存在する様々な制約を満たす必要がある.

\section{問題の記述}
水系運用計画最適化問題の基本要素,制約条件および評価指標を示す.
\subsection{基本要素}
	水系運用計画最適化問題における基本要素を列挙する.
	\subsubsection*{ダム} 
		ダムは放流路への水の放流を行なうものである.
		ダムには貯水機能を持つ通常のダムと持たないものがあり,貯水機能を持たないものを
		特に分流施設と呼ぶこととする.
		通常のダムは上流から流入する水を蓄え,任意の時間帯に放流を行なうことができるが,
		分流施設は上流から流入する水をそのまま下流へと流すこととなる.
	\subsubsection*{発電放流路}
		発電放流路は発電機が存在する放流路である.
		この放流路に放流がなされることによって,発電が行なわれる.
	\subsubsection*{ゲート放流路}
		ダムからの溢水を防ぐための放流を行なう放流路である.
	\subsubsection*{バイパス放流路}
		主に周囲の環境の維持を目的として放流が行われる放流路である.	
	\subsubsection*{スイッチ放流路}
		特定の発電機群の運転状況に応じて放流が行なわれる放流路である.
		発電機群が運転していれば放流が行なわれる放流路と
		停止していれば放流が行なわれる放流路の2つが存在する.
	\subsubsection*{特別放流路}
		以上の放流路のいずれにも属さず,発電・ゲート放流量を決定することにより,
		放流量が定まる放流路である.
\subsection{制約条件}
	水系運用計画最適化問題における制約条件を要素ごとにまとめる.
	\subsubsection*{ダム} 
		\begin{itemize}
			\item 貯水量制約 \\
			 ダムの貯水量が許容範囲内になければならない.
			\item 最終貯水量制約 \\
			 ダムの貯水量の最終値が許容範囲内になければならない.
		\end{itemize}	
	\subsubsection*{発電放流路}
		\begin{itemize}
			\item 水量制約 \\
			 発電機の運転時には使用水量が許容範囲内になければならない.
			\item 電水比制約 \\
			 電水比により,発電機の使用水量と発電電力量が規定される.
			\item 計画運転・停止制約 \\
			 計画運転・停止期間において発電機は運転,%
			または停止状態になければならない.
			\item 短時間運転・停止制約 \\
			 短時間の間に発電機の運転と停止を切り替えるのは望ましくなく,
			停止している発電機を起動すると,一定時間は運転を継続する必要がある.
			また,発電機を停止すると,起動には一定時間間隔を空ける必要がある.
			\item ALL制約 \\
			 周囲の環境への影響を抑えるため,%
			発電所(特定の発電機の組)に使用する水量は%
			段階的に上昇させなければならない.
			\item 夜間出力増制約 \\
			 発電機の出力増加には騒音が伴うため,%
			夜間の時間帯では発電機の出力を増加させてはならない.
		\end{itemize}
	\subsubsection*{スイッチ放流路}
		\begin{itemize}
			\item 分流制約 \\
			 スイッチ放流路は特定の発電機群の運転状況によって,%
			放流を行なうかどうかが決まる.発電機群が運転しているときに放流を行なう
			放流路と停止しているときに放流を行なう放流路の2つがある.
		\end{itemize}	
	\subsubsection*{ゲート放流路,スイッチ放流路,特別放流路}
		\begin{itemize}
			\item 放流量制約 \\
			 すべての放流路について,放流量は許容範囲内になければならない.
		\end{itemize}
	
	\subsection{評価指標}
		時間帯ごとの発電価値の総和最大化とゲート放流量の総和最小化を評価指標とする.
		
\section{まとめ}
本章では,水系運用計画最適化問題の概要を述べ,問題の基本要素,制約条件ならびに評価指標%
について記述した.次章では,本章での記述に基づいた数理計画モデルを示す.