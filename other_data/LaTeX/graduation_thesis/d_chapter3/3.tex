% 第3章
%!TEX root = main.tex

本章では,機体を単一の剛体とみなしたときの6自由度(3軸方向,3軸まわり回転)非線形運動方程式,および3つの姿勢角(オイラー角)の微分方程式による,計9次の非線形微分方程式を記述し,特に縦運動(前後,上下,機首の上下回転運動)にのみ着目し,飛行モデルを設定する.さらに,微小運動を仮定した非線形モデルの線形化を行なう.最後に,機体まわりに働く空気力のモデルの設定について述べる.

%%%%%%%%%%%%%%%%%%%%%%
\section{座標系の導入}
%%%%%%%%%%%%%%%%%%%%%%

\ref{}に示すように,機体に固定した回転座標系$a^{(B)}-x_B,y_B,z_B$(機体座標系)を導入する.また,機体の姿勢を定義するため,\ref{}に示すように,地球に固定した$a^{(E)}-x_E,y_E,z_E$(地球固定座標系)を導入する.

機体姿勢は,姿勢角$\psi,\theta,\phi$によって表し,それぞれヨー角,ピッチ角,ロール角である.これらはオイラー角とよばれ,この順序で機体を回転させることにより,座標系の変換を行なう.

\subsection{座標系の変換}

例えば,$a^{(E)}$から$a^{(B)}$への座標変換行列を$A^{(B,E)}$と書くとすると
\begin{equation}
  \left[
    \begin{array}{ccc}
      x \\
      y \\
      z
    \end{array}
  \right]_B =
  A^{(B,E)}
  \left[
    \begin{array}{ccc}
      x \\
      y \\
      z
    \end{array}
  \right]_E
\end{equation}
となり
\begin{equation}
  \begin{split}
    A^{(B,E)} &=
    \left[
      \begin{array}{ccc}
        1 & 0 & 0 \\
        0 & c\phi & s\phi \\
        0 & -s\phi & c\phi
      \end{array}
    \right]
    \left[
      \begin{array}{ccc}
        c\theta & 0 & -s\theta \\
        0 & 1 & 0 \\
        s\theta & 0 & c\theta
      \end{array}
    \right]
    \left[
      \begin{array}{ccc}
        c\psi & s\psi & 0 \\
        -s\psi & c\psi & 0 \\
        0 & 0 & 1
      \end{array}
    \right] \\
    &=
    \left[
  	\begin{array}{ccc}
    	c\theta c\psi & c\theta s\psi & -s\theta \\
    	s\phi s\theta c\psi - c\phi s\psi & s\phi s\theta s\psi + c\phi c\psi & s\phi c\theta \\
    	c\theta s\theta c\psi + s\phi s\psi & c\phi s\theta s\psi - s\phi c\psi & c\phi c\theta
  	\end{array}
  	\right]
  \end{split}
\end{equation}
のように表される.簡略のため,$\sin* = s*$,$\cos* = c*$と表記している.

また,$A^{(B,E)}$は直交行列であり,逆行列は転置行列に等しいため
\begin{equation}
  A^{(E,B)} =
  \left[
  \begin{array}{ccc}
    c\theta c\psi & s\phi s\theta c\psi - c\phi s\psi & c\theta s\theta c\psi + s\phi s\psi \\
    c\theta s\psi & s\phi s\theta s\psi + c\phi c\psi & c\phi s\theta s\psi - s\phi c\psi \\
    -s\theta & s\phi c\theta & c\phi c\theta
  \end{array}
  \right]
\end{equation}
となる.

\subsection{オイラー角の微分方程式}

オイラー角と機体角速度の関係は
\begin{equation}
  \left[
  \begin{array}{ccc}
    \dot{\phi} \\
    \dot{\theta} \\
    \dot{\psi}
  \end{array}
  \right]
   =
  \left[
  \begin{array}{ccc}
    1 & \sin\phi\tan\theta & \cos\phi\tan\theta \\
    0 & \cos\phi & -\sin\phi \\
    0 & \frac{\sin\phi}{\cos\theta} & \frac{\cos\phi}{\cos\theta}
  \end{array}
  \right]
  \left[
  \begin{array}{ccc}
    p \\
    q \\
    r
  \end{array}
  \right]
\end{equation}
となり,これはオイラー角のキネマティクス方程式とよばれる.ここで,$\dot{\phi},\dot{\theta},\dot{\psi}$はそれぞれ姿勢角の時間微分であり,$p,q,r$はそれぞれ機体座標系における角速度の$x,y,z$成分である.

%%%%%%%%%%%%%%%%%%%%%%
\section{非線形モデル}
%%%%%%%%%%%%%%%%%%%%%%

並進運動と回転運動について,それぞれ運動方程式を記述する.その後,回転翼機モードでの低速飛行における縦運動の運動方程式についてまとめ,機体の非線形モデルとして設定する.

\subsection{非線形並進運動方程式}

機体座標系で表した並進運動方程式は,機体重量を$m$として
\begin{equation}
  \left[
  \begin{array}{ccc}
    \dot{u} \\
    \dot{v} \\
    \dot{w}
  \end{array}
  \right]
   =
  \left[
  \begin{array}{rrr}
    0 & r & -q \\
    -r & 0 & p \\
    q & -p & 0
  \end{array}
  \right]
  \left[
  \begin{array}{ccc}
    u \\
    v \\
    w
  \end{array}
  \right] + \dfrac{1}{m}
  \left[
  \begin{array}{ccc}
    F_x \\
    F_y \\
    F_z
  \end{array}
  \right]
  \label{mot_eq}
\end{equation}
where
\begin{equation*}
  \mbox{\boldmath $V$} =
  \left[
  \begin{array}{c}
    u \quad v \quad w
  \end{array}
  \right]^{\mathrm{T}} :
  \mbox{機体の速度}
\end{equation*}
\begin{equation*}
  \mbox{\boldmath $F$} =
  \left[
  \begin{array}{c}
    F_x \quad F_y \quad F_z
  \end{array}
  \right]^{\mathrm{T}} :
  \mbox{機体に働く力}
\end{equation*}
となる.式(\ref{mot_eq})の右辺第2項の力は,重力,空気力,ロータ推力に分けて
\begin{equation}
    \mbox{\boldmath $F$} = \mbox{\boldmath $F_g$} + \mbox{\boldmath $F_a$} + \mbox{\boldmath $F_t$}
\end{equation}
where
\begin{align*}
  \mbox{\boldmath $F_g$} =
  \left[
  \begin{array}{c}
    X_g \quad Y_g \quad Z_g
  \end{array}
  \right]^{\mathrm{T}},
  \mbox{\boldmath $F_a$} =
  \left[
  \begin{array}{c}
    X_a \quad Y_a \quad Z_a
  \end{array}
  \right]^{\mathrm{T}},
  \mbox{\boldmath $F_t$} =
  \left[
  \begin{array}{c}
    X_t \quad Y_t \quad Z_t
  \end{array}
  \right]^{\mathrm{T}}
\end{align*}
と書ける.ここで重力について
\begin{equation}
  \mbox{\boldmath $F_g$} \triangleq
  \left[
  \begin{array}{ccc}
    X_g \\
    Y_g \\
    Z_g
  \end{array}
  \right] =
  A^{(B,E)}
  \left[
  \begin{array}{ccc}
    0 \\
    0 \\
    mg
  \end{array}
  \right] = mg
  \left[
  \begin{array}{ccc}
    -\sin\theta \\
    \sin\phi\cos\theta \\
    \cos\phi\cos\theta
  \end{array}
  \right]
\end{equation}
である.またロータ推力について,$y$軸方向の力$Y_t=0$であるから,ティルト角を$\gamma$,メインロータ推力を$T_m$,右左サブロータ推力を$T_r$,機首サブロータ推力を$T_f$とすれば次のようになる.
\begin{equation}
  \mbox{\boldmath $F_t$} \triangleq
  \left[
    \begin{array}{ccc}
      X_t \\
      Y_t \\
      Z_t
    \end{array}
  \right] =
  \left[
    \begin{array}{ccc}
      -\cos\gamma & 0 & \sin\gamma \\
      0 & 0 & 0 \\
      -\sin\gamma & 0 & -\cos\gamma
    \end{array}
  \right]
  \left[
    \begin{array}{ccc}
      0 \\
      0 \\
      T_m
    \end{array}
  \right] -
  \left[
    \begin{array}{ccc}
      0 \\
      0 \\
      T_r + T_f
    \end{array}
  \right]
\end{equation}

空気力項については,\ref{}で詳しく述べる.

% \begin{equation}
%   \left[
%   \begin{array}{ccc}
%     \dot{u} \\
%     \dot{v} \\
%     \dot{w}
%   \end{array}
%   \right]
%    =
%   \left[
%   \begin{array}{rrr}
%     0 & r & -q \\
%     -r & 0 & p \\
%     q & -p & 0
%   \end{array}
%   \right]
%   \left[
%   \begin{array}{ccc}
%     u \\
%     v \\
%     w
%   \end{array}
%   \right] + g
%   \left[
%   \begin{array}{ccc}
%     -\sin\theta \\
%     \sin\phi\cos\theta \\
%     \cos\phi\cos\theta
%   \end{array}
%   \right] + \dfrac{1}{m}
%   \left[
%   \begin{array}{ccc}
%     X_a + X_t \\
%     Y_a + Y_t \\
%     Z_a + Z_t
%   \end{array}
%   \right]
% \end{equation}

\subsection{非線形回転運動方程式}

次に,回転の運動について考える.本研究で開発した機体は,機体座標系における$x$軸と$z$軸で張られる面に対し面対称であるため,慣性乗積を0とすると,慣性行列は次のようになる.
\begin{equation}
  I =
  \left[
  \begin{array}{ccc}
    I_{xx} & 0 & -I_{xz} \\
    0 & I_{yy} & 0 \\
    -I_{xz} & 0 & I_{zz}
  \end{array}
  \right]
\end{equation}
このとき,機体の回転運動方程式は
\begin{equation}
I\left[
  \begin{array}{ccc}
    \dot{p} \\
    \dot{q} \\
    \dot{r}
  \end{array}
  \right] =
  \left[
  \begin{array}{rrr}
    0 & r & -q \\
    -r & 0 & p \\
    q & -p & 0
  \end{array}
  \right]
  I
  \left[
  \begin{array}{ccc}
    p \\
    q \\
    r
  \end{array}
  \right] +
  \left[
  \begin{array}{ccc}
    L \\
    M \\
    N
  \end{array}
  \right]
  \label{roll_eq}
\end{equation}
where
\begin{equation*}
  \mbox{\boldmath $M$} =
  \left[
  \begin{array}{c}
    L \quad M \quad N
  \end{array}
  \right]^{\mathrm{T}} :
  \mbox{各機体軸まわりのモーメント}
\end{equation*}
となる.式(\ref{roll_eq})の右辺第2項のモーメントは,重力,空気力,ロータ推力それぞれによるモーメントに分けて
\begin{equation}
    \mbox{\boldmath $M$} = \mbox{\boldmath $M_g$} + \mbox{\boldmath $M_a$} + \mbox{\boldmath $M_t$}
\end{equation}
where
\begin{align*}
  \mbox{\boldmath $M_g$} =
  \left[
  \begin{array}{c}
    L_g \quad M_g \quad N_g
  \end{array}
  \right]^{\mathrm{T}},
  \mbox{\boldmath $M_a$} =
  \left[
  \begin{array}{c}
    L_a \quad M_a \quad N_a
  \end{array}
  \right]^{\mathrm{T}},
  \mbox{\boldmath $M_t$} =
  \left[
  \begin{array}{c}
    L_t \quad M_t \quad N_t
  \end{array}
  \right]^{\mathrm{T}}
\end{align*}
と書ける.ここで重力について,機体軸原点から重心までの距離ベクトルを\mbox{\boldmath $R_G$}とすれば
\begin{equation}
  \begin{split}
    \mbox{\boldmath $M_g$} \triangleq
    \left[
    \begin{array}{ccc}
      L_g \\
      M_g \\
      N_g
    \end{array}
    \right] =
    \mbox{\boldmath $R_G$} \times
    m\mbox{\boldmath $g$} &=
    \left[
    \begin{array}{ccc}
      R_{G_x} \\
      0 \\
      R_{G_z}
    \end{array}
    \right] \times
    A^{(B,E)}
    \left[
    \begin{array}{ccc}
      0 \\
      0 \\
      mg
    \end{array}
    \right] \\
    &= mg
    \left[
    \begin{array}{ccc}
      R_{G_z}\cos\theta\sin\phi \\
      -R_{G_z}\sin\theta - R_{G_x}\cos\theta\cos\phi \\
      R_{G_x}\cos\theta\sin\phi
    \end{array}
    \right]
  \end{split}
\end{equation}
である.また,ロータ推力によるモーメントは,\ref{}によるとまだ未解明な点も多いため,本研究で扱う縦運動に関係するピッチングモーメント$M_t$のみ記載することにする.機体軸原点から各ロータまでの距離を\ref{}のように定めると
\begin{equation}
  M_t = -l_m T_m \cos\gamma + l_f T_f - l_{r_y} T_r
\end{equation}

空気力項については,\ref{}で詳しく述べる.

%%%%%%%%%%%%%%%%%%%%%%
\section{空気力モデル}
%%%%%%%%%%%%%%%%%%%%%%



%%%%%%%%%%%%%%%%%%%%
\section{線形モデル}
%%%%%%%%%%%%%%%%%%%%
