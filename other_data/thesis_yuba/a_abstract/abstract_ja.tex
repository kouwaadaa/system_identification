% 要旨
%!TEX root = main.tex

\begin{titlepage}
%%% ページ番号つける?そのままだと番号がつきます.
%\thispagestyle{empty}
\begin{center}
{\LARGE \bf ティルトロータ型UAVにおける\\低速飛行特性の解析}
\\[0.5cm]
{\Large 弓場~洋輝}
\\[1.0cm]
{\LARGE \bf 要\vspace{36pt}   旨\\}
\end{center}

本研究では,大規模災害発生時における情報収集や物資運搬を目的としたティルトロータ型UAV(Unmanned Arial Vehicle)の回転翼機モードにおける低速飛行モデルの構築と飛行特性の解析を行なう.対象とするUAVは同軸二重反転構造のメインロータを有しており,このメインロータをティルトさせることにより,回転翼機モード・遷移モード・固定翼機モードの3つの飛行モードの切り替えを行なう.回転翼機モードでは,機体中央部のメインロータだけでなく,機体の左右翼端と機首部に設置したサブロータを用いて,浮上揚力と姿勢制御トルクを発生させて飛行する.

はじめに,回転翼機モードを対象とし,低速飛行時の空気力を考慮した力学モデルを導出する.特に,機体の縦方向の運動にのみ着目し,低次元化したモデルを構築する.次に研究対象機を用いて,マニュアル操縦での飛行実験を行ない,得られた飛行データからパラメータ同定を行なう手法を述べた上で,実際の実験結果を示す.最後に,機体運動の固有値解析について述べた後,同定したパラメータをもとに,仮定したモデルの妥当性の検証を行なう.同定結果については,CFD(Computational Fluid Dynamics)の解析結果とも比較し考察を行なう.

\end{titlepage}
